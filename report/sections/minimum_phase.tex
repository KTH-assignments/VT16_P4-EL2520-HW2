For the minimum phase case, the Relative Gain Array suggests that inputs $u_1$
and $u_2$ should be paired to outputs $y_1$ and $y_2$ respectively. The
controller $F(s)$ will thus be of the form:

\[
F(s)=
\begin{bmatrix}
  f_1(s) & 0      \\
  0      & f_2(s) \\
\end{bmatrix}
\]
where $f_i = K_i (1 + \dfrac{1}{T_i s})$, $i \in \{1,2\}$. Here,
$$T_i = \dfrac{1}{\omega_c^{mp}} tan(\phi_m - \pi/2 - arg(G_{ii}(j\omega_c^{mp})))$$
and
$$K_i = \dfrac{1}{\Big|G_{ii}(j\omega_c^{mp}) (1 + \dfrac{1}{j\omega_c^{mp}T_i})\Big|}$$
where $\omega_c^{mp} = 0.1$ rad/s and $\phi_m = 60^{\circ}$. From the above two
relations we conclude that

\begin{align*}
  f_1(s) &= 1.6776 (1 + \dfrac{1}{5.9037s})  \\
  f_2(s) &= 2.0137 (1 + \dfrac{1}{6.3911s})  \\
\end{align*}

Figure \ref{fig:minimum_phase_step_23} shows the step response of the system
for an input step signal applied at input $u_1$ at $t=100$s and at input
$u_2$ at $t=500$s.

\begin{figure}[H]\centering
  \scalebox{0.8}{% This file was created by matlab2tikz.
%
%The latest updates can be retrieved from
%  http://www.mathworks.com/matlabcentral/fileexchange/22022-matlab2tikz-matlab2tikz
%where you can also make suggestions and rate matlab2tikz.
%
\definecolor{mycolor1}{rgb}{0.00000,0.44700,0.74100}%
\definecolor{mycolor2}{rgb}{0.85000,0.32500,0.09800}%
%
\begin{tikzpicture}

\begin{axis}[%
width=4.133in,
height=3.26in,
at={(0.693in,0.44in)},
scale only axis,
xmin=0,
xmax=1000,
xlabel={Time [sec]},
xmajorgrids,
ymin=0,
ymax=1.2,
ylabel={Amplitude},
ymajorgrids,
axis background/.style={fill=white},
legend style={at={(0.850,0.55)},anchor=south west,legend cell align=left,align=left,draw=white!15!black}
]
\addplot [color=mycolor1,solid]
  table[row sep=crcr]{%
0	0\\
20	0\\
40	0\\
60	0\\
80	0\\
99.9999999999991	0\\
100	0\\
100.000000000001	5.30953454319944e-14\\
100.250000000001	0.0146918265802209\\
100.500000000001	0.0295681572278125\\
100.754352269814	0.0448780740244144\\
101.008654905737	0.060346420923314\\
101.262939066233	0.0759606257755576\\
101.517236527193	0.0917085518396901\\
101.771578158086	0.107578386085723\\
102.02599402297	0.123558626431978\\
102.280513450245	0.139638068513165\\
102.538233539972	0.15600104340496\\
102.80124159842	0.172770686293184\\
103.069206362017	0.189916216300979\\
103.341844730837	0.207408553916566\\
103.618911420881	0.225219838836552\\
103.900192586754	0.243323175504045\\
104.185500861003	0.261692449525639\\
104.474671432244	0.280302194593772\\
104.767558903904	0.299127496200249\\
105.064034753733	0.318143922754355\\
105.363985256162	0.337327476907453\\
105.66730976422	0.356654561751558\\
105.973919286369	0.376101958771575\\
106.283735290694	0.395646814046598\\
106.596688711554	0.415266631762208\\
106.912719101436	0.434939271926882\\
107.231773926743	0.454642952634335\\
107.553807956833	0.4743562540475\\
107.878782767495	0.494058125694486\\
108.206666311274	0.513727894350593\\
108.537432563849	0.533345273263562\\
108.871061233585	0.552890372065712\\
109.207537520834	0.572343706683514\\
109.546851936236	0.591686209859247\\
109.889000159453	0.610899241234733\\
110.233086944427	0.629915488373025\\
110.578748776097	0.64869983425306\\
110.925991483882	0.667238507489986\\
111.274822846482	0.685518294748208\\
111.625252016535	0.703526512296904\\
111.977289495889	0.721251007342316\\
112.330947141168	0.738680160016005\\
112.686238156886	0.755802883827925\\
113.043177100722	0.772608625889709\\
113.401779906443	0.789087367127299\\
113.762063902738	0.805229621441023\\
114.124047847578	0.821026434745273\\
114.487751967844	0.836469383387437\\
114.853198003861	0.851550571940727\\
115.220409267144	0.866262630710268\\
115.589410694445	0.880598712248616\\
115.960228929194	0.894552488116609\\
116.332892387295	0.908118144171968\\
116.707431357797	0.921290376322652\\
117.083878088263	0.934064384833462\\
117.46226690676	0.946435869047752\\
117.842634329426	0.958401020700658\\
118.225019204296	0.969956517521459\\
118.609462857332	0.981099515923301\\
118.996009255359	0.991827643244754\\
119.384705182591	1.00213898940327\\
119.775600454867	1.0120320985392\\
120.168748131988	1.02150595950881\\
120.564204775838	1.03055999642639\\
120.96203072752	1.03919405843831\\
121.362290413504	1.04740840891976\\
121.765052697718	1.05520371433682\\
122.170391277785	1.06258103255229\\
122.578385119061	1.06954180032368\\
122.989118953947	1.07608782034962\\
123.402683855753	1.08222124774953\\
123.819177869959	1.08794457547783\\
124.238739901066	1.09326102172145\\
124.661494661109	1.09817372071711\\
125.087566495273	1.10268595806427\\
125.517090614847	1.10680129603561\\
125.950214327985	1.11052355248773\\
126.38709858589	1.1138567792333\\
126.827919814671	1.11680523840386\\
127.272872088309	1.11937337624335\\
127.722169727122	1.12156579374548\\
128.176050436591	1.12338721336681\\
128.634779116461	1.12484244068775\\
129.098652551211	1.12593631968206\\
129.56800521027	1.12667367954967\\
130.043216574457	1.12705927055671\\
130.524720445586	1.12709768490705\\
131.012041251309	1.12679419782256\\
131.502419252492	1.12616009588051\\
131.996184503254	1.12520757204656\\
132.493660871516	1.12394844916943\\
132.995196609959	1.12239417079215\\
133.501168955652	1.12055579345588\\
134.011990016347	1.11844397060294\\
134.528113993576	1.11606892662074\\
135.050046184668	1.11344041848683\\
135.574134212094	1.11059142217341\\
136.093145397752	1.10758202838598\\
136.607736128933	1.1044316139091\\
137.11855469871	1.10115723761079\\
137.626158154405	1.09777449935253\\
138.131031937983	1.09429775560028\\
138.633605235531	1.09074028280646\\
139.134262427925	1.08711440859724\\
139.63335183163	1.08343161879478\\
140.131192526056	1.07970264581688\\
140.62807973056	1.07593754302648\\
141.12428914805	1.0721457476813\\
141.62008049684	1.06833613502695\\
142.115700451852	1.06451706500247\\
142.611385111782	1.06069642304981\\
143.107362116116	1.0568816559108\\
143.60385247046	1.05307980339128\\
144.101072164276	1.04929752657457\\
144.599233603549	1.04554113321231\\
145.09854695656	1.04181660027861\\
145.599221347461	1.03812959478236\\
146.101466048514	1.03448549220637\\
146.605491586262	1.03088939363886\\
147.111510856557	1.02734614126479\\
147.61974023091	1.02386033267952\\
148.130400717853	1.02043633388139\\
148.643719110627	1.01707829169215\\
149.159929286862	1.01379014475537\\
149.676882278416	1.0105902147096\\
150.192029715355	1.0074972371572\\
150.705674946622	1.00451127688307\\
151.218110964667	1.00163224484896\\
151.729612284364	0.998859969754469\\
152.240438256729	0.996194195943857\\
152.750835884986	0.99363458293283\\
153.261042231826	0.991180706234195\\
153.771286482122	0.988832059238359\\
154.281791773973	0.986588055664105\\
154.792776806016	0.984448032604701\\
155.304457330256	0.982411253731779\\
155.817047550568	0.980476912661935\\
156.330761368292	0.978644136743394\\
156.845813685711	0.976911990501752\\
157.362421614688	0.975279479403473\\
157.880805744034	0.97374555343855\\
158.401191409305	0.972309110800673\\
158.923810028875	0.970969001528768\\
159.448900535389	0.969724031126672\\
159.976710930054	0.968572964197921\\
160.507499967308	0.967514528182183\\
161.041539063452	0.966547417111484\\
161.579114471222	0.965670295484672\\
162.120529699398	0.964881802459806\\
162.666108404097	0.964180556188848\\
163.216197725469	0.963565158630355\\
163.771172268328	0.963034200864928\\
164.331438856894	0.962586269152593\\
164.897442317475	0.962219951944216\\
165.469672587087	0.961933848197173\\
166.048673547782	0.961726577467109\\
166.63505414639	0.961596792415314\\
167.229502704293	0.961543194626416\\
167.832805469961	0.961564555117253\\
168.445871333776	0.961659741501222\\
169.06976527817	0.96182775489505\\
169.705755049033	0.962067781471558\\
170.355377867907	0.962379266659881\\
171.020539566722	0.962762025916382\\
171.70366795572	0.963216417206393\\
172.407963060357	0.963743624065634\\
173.137833779374	0.964346152183636\\
173.899730634716	0.965028780839713\\
174.685161035718	0.965782059358849\\
175.462621438348	0.966569621096909\\
176.233710249371	0.967384766812515\\
177.000171825469	0.968221968604393\\
177.763435573061	0.96907619772778\\
178.524704886545	0.969942842129197\\
179.285021260482	0.970817650877084\\
180.045308517652	0.97169669156362\\
180.806404462281	0.972576316837087\\
181.569084544967	0.973453137809994\\
182.3274867653	0.974316544026297\\
183.071460646514	0.975151876668788\\
183.804026887023	0.975960174494189\\
184.527678256927	0.976742224792788\\
185.24431378944	0.977498412543914\\
185.955435417096	0.978228894671884\\
186.662270856609	0.978933703985906\\
187.365851776875	0.979612810982476\\
188.067066040969	0.980266162016455\\
188.766693977737	0.980893703411732\\
189.465434585442	0.981495397041327\\
190.163925090703	0.98207123047138\\
190.862755869987	0.982621223435343\\
191.562482367892	0.983145432018772\\
192.263634759186	0.983643951145083\\
192.966726013436	0.984116915881514\\
193.6722591036	0.984564502083202\\
194.38073322957	0.984986926270873\\
195.092649944537	0.985384445304755\\
195.80851889852	0.98575735562056\\
196.528863729987	0.986105992310077\\
197.25422824027	0.986430728044788\\
197.985183112037	0.986731971893831\\
198.72233333552	0.98701016802063\\
199.466326874119	0.987265794363144\\
200.217864775779	0.987499361217197\\
200.977713359995	0.987711409766634\\
201.747724343934	0.987902745386337\\
202.52897379725	0.988073906392876\\
203.322605764132	0.988225440813976\\
204.129934829566	0.988357924775433\\
204.95249119814	0.988471960375365\\
205.792085107313	0.988568173462356\\
206.650898457863	0.988647210595237\\
207.531619990962	0.988709734647272\\
208.437653558656	0.988756418025777\\
209.373456076334	0.988787931418712\\
210.345123628299	0.988804923604634\\
211.361501211961	0.988807981692671\\
212.436559567364	0.988797542582229\\
213.595521211185	0.988773656379652\\
214.773708615059	0.98873918138855\\
215.936234479576	0.988698122169246\\
217.0825546228	0.988653462509813\\
218.218391188015	0.988607461509934\\
219.34814126642	0.988562020393153\\
220.472354085465	0.988518884289527\\
221.560919348349	0.988480543341501\\
222.624929397885	0.988447486575862\\
223.67246833532	0.988420135677394\\
224.708946366815	0.988398884583999\\
225.738303979045	0.988384082217534\\
226.763615863341	0.988376028764092\\
227.787414925199	0.988374977228692\\
228.811884708465	0.988381136726161\\
229.838983344675	0.988394676304686\\
230.870530033531	0.988415728895923\\
231.908269738158	0.988444395282756\\
232.953926036136	0.988480748112689\\
234.009248041875	0.98852483604843\\
235.076056017969	0.988576688201792\\
236.156289872289	0.988636319067854\\
237.252064674781	0.988703734252879\\
238.365738524414	0.988778937430937\\
239.500000813181	0.988861939237124\\
240.65799199316	0.988952769151355\\
241.843474408922	0.989051492221802\\
243.061088181209	0.98915823390591\\
244.316754857839	0.989273219143062\\
245.61835780307	0.989396838298954\\
246.976985017293	0.989529768203087\\
248.409468659076	0.989673221238976\\
249.944476955836	0.989829548009343\\
251.641595050238	0.990004142385747\\
253.357610742345	0.990181143285801\\
255.028634373048	0.99035270888576\\
256.67343657809	0.990519796137848\\
258.290558338346	0.990681529426442\\
259.850436823022	0.990834535679718\\
261.372502603101	0.990980541092769\\
262.868706170972	0.991120574147828\\
264.350868429596	0.991255657221228\\
265.827331379503	0.991386481567073\\
267.304562725912	0.991513566488836\\
268.788023633886	0.991637342348212\\
270.282661101264	0.991758195163394\\
271.793233206649	0.991876493721183\\
273.324563665081	0.991992608532474\\
274.881777385033	0.992106927270159\\
276.470557954096	0.992219869853715\\
278.097473269055	0.992331906173656\\
279.770430170909	0.992443579863916\\
281.499377124838	0.992555544610018\\
283.29748534225	0.992668625134467\\
285.18334078427	0.992783930929516\\
287.185569746753	0.992903097886003\\
289.354599478096	0.993028906914714\\
291.803978887809	0.993167472704281\\
294.263800014221	0.993303477235385\\
296.66839230737	0.99343385667296\\
299.035421199937	0.993560073659793\\
301.296072393802	0.993678886537599\\
303.509810823309	0.993793744431099\\
305.711158675424	0.993906586336364\\
307.922226984074	0.994018590572946\\
310.160827507969	0.994130631374876\\
312.443878544101	0.994243465184732\\
314.789643153763	0.994357842985134\\
317.220103439154	0.994474616475744\\
319.764607744235	0.994594888919801\\
322.467152727765	0.994720307626923\\
325.405025527868	0.994853811263801\\
328.758020652587	0.995002430566729\\
332.359608637169	0.995157491503098\\
335.806746726032	0.995301398533328\\
338.970094475205	0.995429581068323\\
342.018673729022	0.995549641060118\\
345.044507864388	0.995665500841919\\
348.094542406307	0.995779041810487\\
351.204163159182	0.995891551218613\\
354.407253993914	0.996004134060409\\
357.741833806038	0.996117911089648\\
361.256197956122	0.996234196579337\\
365.019226547411	0.996354765751126\\
369.143518605252	0.996482437450577\\
373.856881659547	0.996622885993997\\
378.637321020323	0.996759661976612\\
383.062677641711	0.996881406815074\\
387.429691027822	0.996997131324116\\
391.878330359897	0.997110667167044\\
396.500356685545	0.997224135056101\\
401.389255205443	0.997339346240872\\
406.66043311064	0.99745823602904\\
412.049581648717	0.997574300475696\\
417.228072131693	0.997680824762932\\
422.347392309865	0.99778152143671\\
427.503506701113	0.997878511044392\\
432.755895330962	0.997972943243067\\
438.145378874319	0.998065466076045\\
443.696909866936	0.99815635450205\\
449.417123346912	0.998245540276334\\
455.291823307406	0.99833264844556\\
461.288881611296	0.998417114896147\\
467.369255747492	0.998498390494648\\
473.501154326418	0.998576131408131\\
479.668439238791	0.998650263047829\\
485.870099918996	0.998720916892847\\
492.114699306059	0.998788323349485\\
498.414532111917	0.998852726729879\\
499.999999999996	0.998868388475261\\
500	0.998868388475261\\
500.000000000004	0.998868388475261\\
500.31834636312	0.99891871472164\\
500.63197508892	0.999059654062116\\
500.94186526661	0.999286465308419\\
501.248022346767	0.999594408354884\\
501.550429943247	0.999978745101005\\
501.849060964534	1.00043474936608\\
502.143877431601	1.00095771223078\\
502.434829998634	1.00154294586033\\
502.721857248483	1.00218578570771\\
503.004884783965	1.00288159103595\\
503.283824027904	1.00362574346356\\
503.558570695393	1.00441364327819\\
503.829002857205	1.00524070311743\\
504.094978462369	1.0061023383888\\
504.356332173736	1.00699395364888\\
504.61633241286	1.00792357107638\\
504.876756037587	1.0088957082393\\
505.137624040384	1.00990901544851\\
505.398961280752	1.01096213597477\\
505.660792649272	1.01205369343045\\
505.923143100671	1.01318229265115\\
506.186037720992	1.0143465207099\\
506.4495017252	1.0155449476145\\
506.713560522834	1.01677612727889\\
506.978239739937	1.01803859828384\\
507.2435652605	1.01933088470347\\
507.509563267939	1.02065149691338\\
507.776260274683	1.02199893231816\\
508.043683163986	1.02337167611459\\
508.311859241729	1.02476820208466\\
508.58081626006	1.02618697321863\\
508.850582471344	1.02762644246697\\
509.121186673108	1.02908505341974\\
509.392658252601	1.03056124095427\\
509.665027235827	1.03205343187518\\
509.938324351135	1.0335600456017\\
510.212581058258	1.03507949462944\\
510.487829632323	1.03661018525645\\
510.76410321284	1.03815051807628\\
511.041435875279	1.03969888855193\\
511.319862695145	1.04125368750048\\
511.599419820154	1.04281330157164\\
511.880144564626	1.04437611379453\\
512.162075479906	1.04594050392978\\
512.447281520475	1.04751602861864\\
512.735769205064	1.04910066308955\\
513.027473806076	1.05069191224402\\
513.322339582465	1.05228726239863\\
513.62031868974	1.05388418067764\\
513.921370725437	1.05548011759486\\
514.225462377318	1.0570725098733\\
514.532158048293	1.05865668569302\\
514.839113665298	1.06021843255452\\
515.146377641458	1.06175618975759\\
515.453994969585	1.06326842121723\\
515.762009997539	1.06475362903412\\
516.070466537511	1.06621035326152\\
516.379407987427	1.06763717178072\\
516.688877428121	1.06903270010127\\
516.998917728687	1.07039559123414\\
517.309571639308	1.07172453553449\\
517.620881890773	1.07301826059414\\
517.932891287949	1.0742755311204\\
518.24564279878	1.07549514880533\\
518.559179643238	1.07667595220056\\
518.873545381452	1.07781681659004\\
519.188784016366	1.0789166539099\\
519.504940069238	1.07997441256118\\
519.822058692538	1.0809890773417\\
520.140185739851	1.0819596692176\\
520.459367894441	1.08288524526511\\
520.779652753844	1.08376489844581\\
521.101088938615	1.08459775743347\\
521.42372621195	1.08538298643596\\
521.747615593556	1.08611978496509\\
522.072809487037	1.08680738760194\\
522.399361817185	1.0874450637389\\
522.727328170532	1.0880321172796\\
523.056765951483	1.08856788631679\\
523.387734542337	1.08905174276012\\
523.720295496728	1.08948309195215\\
524.054512729388	1.08986137220335\\
524.390452726196	1.09018605428127\\
524.728184779565	1.0904566408453\\
525.0677812576	1.09067266581571\\
525.409317882281	1.09083369364352\\
525.752874042345	1.09093931849454\\
526.098533146299	1.09098916332854\\
526.446383023346	1.09098287885078\\
526.796516352735	1.09092014231445\\
527.149031160491	1.09080065615754\\
527.504031382224	1.09062414643719\\
527.861627502721	1.09039036102386\\
528.221937271471	1.09009906752338\\
528.585086541503	1.08975005085512\\
528.951210221038	1.08934311043772\\
529.320453387644	1.08887805688063\\
529.692972573291	1.0883547081\\
530.068937286825	1.08777288469545\\
530.448531796279	1.08713240445324\\
530.831957252729	1.08643307573895\\
531.219434243012	1.08567468948431\\
531.611205873162	1.08485700940483\\
532.007541507803	1.08397975999613\\
532.408741408665	1.0830426115397\\
532.81514251378	1.08204516123527\\
533.22712570976	1.08098690921968\\
533.645125173164	1.07986722751634\\
534.069569410878	1.07868552089559\\
534.489876333892	1.0774735233978\\
534.906481395403	1.07623373843528\\
535.319886938251	1.07496815681072\\
535.730519730943	1.07367871171353\\
536.138746216773	1.07236728622268\\
536.544885121084	1.07103571560572\\
536.949216906997	1.06968578960866\\
537.351990925577	1.06831925487654\\
537.75343096531	1.06693781718013\\
538.153739577841	1.06554314359288\\
538.553101583193	1.06413686425476\\
538.951686817448	1.0627205742504\\
539.349652484142	1.06129583489153\\
539.747144999831	1.05986417520553\\
540.144301603725	1.05842709299461\\
540.54125166842	1.05698605593559\\
540.938117836329	1.05554250246759\\
541.335017012362	1.05409784250818\\
541.732061191602	1.05265345819919\\
542.129358183106	1.0512107045582\\
542.527012311582	1.04977090981895\\
542.925124956556	1.04833537603718\\
543.32379509932	1.04690537939759\\
543.723119771484	1.04548217065165\\
544.123194518928	1.04406697531479\\
544.524113818759	1.04266099388308\\
544.925971433406	1.04126540214957\\
545.328860779638	1.03988135136926\\
545.732875300376	1.03850996834278\\
546.138108785078	1.03715235562424\\
546.544655711612	1.035809591619\\
546.952611568274	1.03448273072933\\
547.362073222318	1.03317280334662\\
547.77313925954	1.03188081595721\\
548.185910288947	1.03060775138199\\
548.600489392474	1.02935456860688\\
549.016982445513	1.02812220307923\\
549.435498582041	1.02691156663866\\
549.856150614358	1.02572354768576\\
550.279055524849	1.02455901125457\\
550.704334998969	1.02341879910992\\
551.132115979452	1.022303729946\\
551.562531319621	1.02121459950457\\
551.995720502807	1.02015218074565\\
552.431830390644	1.01911722419265\\
552.871016186603	1.01811045804432\\
553.313442403006	1.01713258862898\\
553.759284031189	1.01618430077851\\
554.20872788133	1.0152662582946\\
554.661974127868	1.01437910452292\\
555.11923808575	1.0135234630904\\
555.580752334552	1.01269993871333\\
556.046769204917	1.01190911822809\\
556.515720620883	1.01115445678967\\
556.985921847559	1.0104389577316\\
557.457558013575	1.00976236011133\\
557.930810389902	1.00912439959298\\
558.405865933807	1.00852479518712\\
558.882918221224	1.00796325033426\\
559.362168504754	1.00743945394511\\
559.843826923369	1.00695308142133\\
560.328113814574	1.00650379575627\\
560.816201270621	1.00609049020849\\
561.308419421652	1.005712967708\\
561.80502285831	1.00537110016683\\
562.306111398754	1.00506483976799\\
562.810660702875	1.00479463936526\\
563.318904975146	1.00456002612189\\
563.831079935302	1.00436051336518\\
564.347436710021	1.0041955935671\\
564.868243968695	1.00406473925218\\
565.393790536509	1.00396740401398\\
565.924388475383	1.00390302373569\\
566.460376750467	1.00387101806526\\
567.002125628407	1.00387079221289\\
567.55004201467	1.00390173916046\\
568.104575977818	1.00396324240469\\
568.666228869286	1.0040546794065\\
569.23556349188	1.00417542598329\\
569.813217086472	1.00432486199369\\
570.399918155674	1.00450237882257\\
570.996508621679	1.00470738941347\\
571.603973678681	1.004939342023\\
572.223482949026	1.00519773954085\\
572.856448749962	1.00548216737337\\
573.501644763855	1.00579087559593\\
574.137024006327	1.00611104259367\\
574.764145773858	1.00644052981767\\
575.384642535434	1.00677762526473\\
575.999803617571	1.00712074761493\\
576.610673107679	1.00746842444466\\
577.218119073226	1.00781927891229\\
577.822879665671	1.00817201896253\\
578.425594932892	1.00852542846774\\
579.026829571595	1.0088783599551\\
579.627089711959	1.00922972860794\\
580.226835503505	1.00957850720112\\
580.826490929648	1.00992372188468\\
581.426451567414	1.0102644485965\\
582.027090795178	1.01059980994758\\
582.628764962777	1.01092897255143\\
583.231817776277	1.01125114469935\\
583.836584091547	1.01156557430793\\
584.443393211499	1.01187154705427\\
585.05257200654	1.01216838475656\\
585.664447744137	1.01245544384916\\
586.279350818137	1.01273211397739\\
586.897617543546	1.01299781672001\\
587.519592897876	1.01325200432292\\
588.145633393036	1.01349415847459\\
588.776110262748	1.01372378913482\\
589.411412809974	1.01394043329201\\
590.051952218428	1.01414365370259\\
590.698165897278	1.01433303755296\\
591.350522480647	1.01450819499358\\
592.009527757211	1.01466875751844\\
592.675731690004	1.01481437609831\\
593.346232600269	1.01494408522612\\
594.020492179409	1.01505764415938\\
594.698958247129	1.01515505983836\\
595.382085260243	1.01523635957311\\
596.070346342781	1.01530159163921\\
596.764238116862	1.01535082419104\\
597.464286337627	1.01538414404694\\
598.171052250718	1.01540165527494\\
598.885140068764	1.01540347754502\\
599.607205769055	1.01538974418028\\
600.337967872062	1.01536059982063\\
601.078220612266	1.01531619757157\\
601.828850541268	1.01525669545698\\
602.589460674147	1.01518240079271\\
603.359008211947	1.01509380764226\\
604.138396061584	1.01499128132482\\
604.928595769641	1.01487519003718\\
605.730701267599	1.01474589782888\\
606.54595871224	1.01460375905897\\
607.375806945696	1.01444911089523\\
608.221932519015	1.01428226304127\\
609.086346893288	1.01410348319853\\
609.971602157534	1.01391295302378\\
610.8822727509	1.01371043608414\\
611.822377435019	1.01349584634422\\
612.783563594335	1.01327208433745\\
613.723953986774	1.01305024858311\\
614.64836061398	1.01283059039818\\
615.561265581139	1.01261320165281\\
616.465946272445	1.01239829227553\\
617.364923989596	1.01218612977137\\
618.260233270125	1.01197700369623\\
619.153581971464	1.01177120653033\\
620.04645296726	1.01156902284864\\
620.940173069121	1.01137072283309\\
621.835961724106	1.01117655836734\\
622.734967811695	1.01098676050403\\
623.638298735958	1.01080153780722\\
624.54704473215	1.01062107522612\\
625.462300486181	1.01044553326809\\
626.385185861108	1.01027504727975\\
627.316866671018	1.01010972680599\\
628.258577160292	1.00994965486759\\
629.211645342538	1.00979488712398\\
630.177523279443	1.00964545077135\\
631.157824100252	1.00950134314598\\
632.154369910289	1.00936252973853\\
633.166557146571	1.00922928637469\\
634.188586209626	1.0091024382881\\
635.222210878138	1.00898172865874\\
636.269216077767	1.00886689521122\\
637.331565128841	1.00875765422712\\
638.411460454255	1.00865369752486\\
639.511425197274	1.00855468848232\\
640.634414975019	1.00846025674912\\
641.783977687813	1.00836999077193\\
642.964492910317	1.00828342662892\\
644.181548988249	1.00820003045784\\
645.442576059524	1.00811916895166\\
646.753128425805	1.00804033966777\\
648.108273023931	1.0079635843978\\
649.518140419641	1.00788798675269\\
650.98031922093	1.00781325801241\\
652.382354563349	1.00774430577491\\
653.744229773779	1.00767918134452\\
655.08197945345	1.00761642148796\\
656.405460711129	1.00755503332065\\
657.721501941295	1.00749427677601\\
659.035336253604	1.00743356378119\\
660.352065436261	1.00737237106043\\
661.676191626664	1.00731024874635\\
663.011271289277	1.00724682871413\\
664.360804306684	1.0071817783186\\
665.728380089987	1.00711478975749\\
667.117823492944	1.00704557026984\\
668.533358306492	1.00697383200022\\
669.973195665075	1.00689962394851\\
671.427428221109	1.00682346924813\\
672.900513952957	1.00674519126873\\
674.396978433219	1.00666463143155\\
675.92180633806	1.00658162851692\\
677.480697068384	1.00649600542446\\
679.080419437341	1.00640755146142\\
680.72934165776	1.00631599647524\\
682.438290277085	1.0062209693753\\
684.22206928707	1.00612192489635\\
686.102454814173	1.00601799857982\\
688.115060365621	1.00590767153412\\
690.306383309738	1.00578901657087\\
692.380404783726	1.00567846656698\\
694.386099532054	1.00557344114723\\
696.362557481768	1.00547194772905\\
698.331632341184	1.00537294556197\\
700.308461075455	1.00527576665088\\
702.305490986239	1.00517989931991\\
704.33432121914	1.00508489413798\\
706.406897045228	1.00499030866903\\
708.536603663039	1.00489566307927\\
710.711818677589	1.00480157695926\\
712.931029872237	1.00470816527887\\
715.212172479621	1.00461472297621\\
717.576530492177	1.00452046677487\\
720.052823738271	1.00442439351118\\
722.684603899091	1.00432503948782\\
725.54967341424	1.00421984912789\\
728.679176480554	1.00410819498509\\
731.84893541316	1.00399827190269\\
734.849396567784	1.00389693459096\\
737.797198748619	1.00379978411389\\
740.758705311609	1.00370446486295\\
743.722646392544	1.00361126958945\\
746.689735142594	1.00352011913703\\
749.698372424421	1.00342983828727\\
752.782376478684	1.00333950710709\\
755.977543485294	1.00324825672607\\
759.327796180672	1.0031551091984\\
762.894604162248	1.00305876317061\\
766.777799928225	1.00295714124582\\
771.16198933914	1.00284643683182\\
775.327885081216	1.00274511333198\\
779.440383971789	1.00264868155408\\
783.618456608122	1.00255424889399\\
787.943843994626	1.00246010457546\\
792.38178141451	1.00236718241121\\
797.002855303264	1.00227420798751\\
801.903615768175	1.00217963161685\\
807.052899524138	1.00208451360988\\
812.237984740189	1.00199293079435\\
817.485745144042	1.00190432829258\\
822.814351756779	1.00181837628028\\
828.168458256792	1.00173590564804\\
833.644743211414	1.00165540953264\\
839.250685464292	1.00157686517508\\
844.940491573278	1.00150095285656\\
850.705810034808	1.00142776062976\\
856.538391422323	1.00135734850539\\
862.432760405386	1.00128972127443\\
868.386827239582	1.00122483252074\\
874.401459831418	1.00116260014088\\
880.479487435946	1.00110292465942\\
886.624604783139	1.00104570430606\\
892.840474861193	1.0009908444718\\
899.130159511185	1.00093826167835\\
905.495891690834	1.00088788339407\\
911.939132489805	1.00083964538404\\
918.460814444516	1.00079348818188\\
925.061657477152	1.00074935390532\\
931.742454944057	1.00070718413244\\
938.504263857806	1.00066691901825\\
945.348478431896	1.00062849742677\\
952.276805103903	1.00059185765076\\
959.291179949474	1.00055693828431\\
966.393672376379	1.00052367894207\\
973.586408458017	1.00049202068619\\
980.871530672681	1.00046190616356\\
988.251193935047	1.00043327954323\\
995.727587821901	1.00040608636322\\
1000	1.00039132006291\\
};
\addlegendentry{$\text{y}_\text{1}$};

\addplot [color=mycolor2,solid]
  table[row sep=crcr]{%
0	0\\
20	0\\
40	0\\
60	0\\
80	0\\
99.9999999999991	0\\
100	0\\
100.000000000001	3.08509646484951e-28\\
100.250000000001	2.32633136318693e-05\\
100.500000000001	9.28445313341044e-05\\
100.754352269814	0.000210800979701121\\
101.008654905737	0.000375853405746745\\
101.262939066233	0.000587502587959257\\
101.517236527193	0.000845215233829195\\
101.771578158086	0.00114842448618865\\
102.02599402297	0.00149653101535093\\
102.280513450245	0.0018889040534566\\
102.538233539972	0.00233039716538921\\
102.80124159842	0.00282591712205527\\
103.069206362017	0.00337652182888039\\
103.341844730837	0.00398304262876947\\
103.618911420881	0.00464609133124958\\
103.900192586754	0.00536606912125105\\
104.185500861003	0.00614317445310579\\
104.474671432244	0.00697741018486767\\
104.767558903904	0.00786859015242406\\
105.064034753733	0.0088163453465072\\
105.363985256162	0.00982012979505428\\
105.66730976422	0.0108792262195524\\
105.973919286369	0.0119927515598674\\
106.283735290694	0.0131596623774715\\
106.596688711554	0.0143787602402029\\
106.912719101436	0.0156486970294501\\
107.231773926743	0.0169679802839483\\
107.553807956833	0.0183349784759331\\
107.878782767495	0.019747926381523\\
108.206666311274	0.0212049304109351\\
108.537432563849	0.0227039739811952\\
108.871061233585	0.0242429229138754\\
109.207537520834	0.0258195308239453\\
109.546851936236	0.0274314445619749\\
109.889000159453	0.029076209637535\\
110.233086944427	0.0307469060600277\\
110.578748776097	0.0324389834112761\\
110.925991483882	0.0341496122652756\\
111.274822846482	0.0358759336089077\\
111.625252016535	0.0376150614452194\\
111.977289495889	0.039364087959184\\
112.330947141168	0.0411200887333659\\
112.686238156886	0.0428801277980001\\
113.043177100722	0.0446412626336442\\
113.401779906443	0.0464005491486345\\
113.762063902738	0.0481550465181499\\
114.124047847578	0.0499018219754658\\
114.487751967844	0.0516379554976499\\
114.853198003861	0.0533605443771513\\
115.220409267144	0.0550667077108849\\
115.589410694445	0.0567535907193266\\
115.960228929194	0.0584183690315274\\
116.332892387295	0.060058252728597\\
116.707431357797	0.0616704903663795\\
117.083878088263	0.0632523727422292\\
117.46226690676	0.0648012366232966\\
117.842634329426	0.0663144682057072\\
118.225019204296	0.0677895065064902\\
118.609462857332	0.0692238465316192\\
118.996009255359	0.0706150422690981\\
119.384705182591	0.0719607094808345\\
119.775600454867	0.0732585283603887\\
120.168748131988	0.0745062458988736\\
120.564204775838	0.0757016781055816\\
120.96203072752	0.0768427119664258\\
121.362290413504	0.077927307153576\\
121.765052697718	0.0789534975062817\\
122.170391277785	0.0799193922397716\\
122.578385119061	0.0808231768336829\\
122.989118953947	0.0816631136295441\\
123.402683855753	0.0824375421008384\\
123.819177869959	0.0831448787078364\\
124.238739901066	0.0837836639192023\\
124.661494661109	0.0843524615501287\\
125.087566495273	0.0848498911429922\\
125.517090614847	0.08527464386266\\
125.950214327985	0.085625479605193\\
126.38709858589	0.0859012234512229\\
126.827919814671	0.0861007611786986\\
127.272872088309	0.0862230336357542\\
127.722169727122	0.0862670297318752\\
128.176050436591	0.0862317777275742\\
128.634779116461	0.0861163343969831\\
129.098652551211	0.0859197714950919\\
129.56800521027	0.0856411587645446\\
130.043216574457	0.0852795423890111\\
130.524720445586	0.0848339173940777\\
131.012041251309	0.0843043296528387\\
131.502419252492	0.0836944243037006\\
131.996184503254	0.0830052426064247\\
132.493660871516	0.0822378497712199\\
132.995196609959	0.0813932998161978\\
133.501168955652	0.0804726254381751\\
134.011990016347	0.0794768248153636\\
134.528113993576	0.0784068448482121\\
135.050046184668	0.0772635596989611\\
135.574134212094	0.0760576773360946\\
136.093145397752	0.0748104166414071\\
136.607736128933	0.0735256645106044\\
137.11855469871	0.0722068650327062\\
137.626158154405	0.0708572946298817\\
138.131031937983	0.0694800835642986\\
138.633605235531	0.0680782308796455\\
139.134262427925	0.0666546168447549\\
139.63335183163	0.0652120133751828\\
140.131192526056	0.0637530927666607\\
140.62807973056	0.0622804352321728\\
141.12428914805	0.0607965353406132\\
141.62008049684	0.0593038076600351\\
142.115700451852	0.0578045916602377\\
142.611385111782	0.056301156061418\\
143.107362116116	0.0547957026669929\\
143.60385247046	0.0532903698228998\\
144.101072164276	0.0517872354998829\\
144.599233603549	0.0502883201306441\\
145.09854695656	0.048795589068116\\
145.599221347461	0.0473109549958607\\
146.101466048514	0.0458362799509667\\
146.605491586262	0.0443733773121096\\
147.111510856557	0.0429240135573682\\
147.61974023091	0.0414899099246346\\
148.130400717853	0.0400727438689929\\
148.643719110627	0.0386741505844826\\
149.159929286862	0.03729572420354\\
149.676882278416	0.0359451942563566\\
150.192029715355	0.034630684637452\\
150.705674946622	0.0333525379456073\\
151.218110964667	0.0321110482320438\\
151.729612284364	0.0309064883368581\\
152.240438256729	0.0297391065180693\\
152.750835884986	0.0286091240011246\\
153.261042231826	0.0275167332876565\\
153.771286482122	0.0264620971011711\\
154.281791773973	0.0254453477412049\\
154.792776806016	0.0244665868444283\\
155.304457330256	0.0235258853439051\\
155.817047550568	0.0226232836147033\\
156.330761368292	0.0217587919171994\\
156.845813685711	0.0209323907750815\\
157.362421614688	0.020144031584132\\
157.880805744034	0.0193936372091916\\
158.401191409305	0.0186811026915821\\
158.923810028875	0.0180062959936243\\
159.448900535389	0.0173690587779036\\
159.976710930054	0.0167692072288158\\
160.507499967308	0.0162065329504552\\
161.041539063452	0.0156808038864144\\
161.579114471222	0.0151917652952865\\
162.120529699398	0.0147391408745829\\
162.666108404097	0.0143226339148331\\
163.216197725469	0.0139419286400785\\
163.771172268328	0.0135966917140311\\
164.331438856894	0.0132865740017478\\
164.897442317475	0.0130112126449428\\
165.469672587087	0.0127702335674443\\
166.048673547782	0.0125632545698504\\
166.63505414639	0.0123898892275057\\
167.229502704293	0.0122497518691781\\
167.832805469961	0.0121424641281299\\
168.445871333776	0.0120676636881278\\
169.06976527817	0.0120250162852924\\
169.705755049033	0.0120142325727725\\
170.355377867907	0.0120350925682394\\
171.020539566722	0.0120874823063886\\
171.70366795572	0.0121714511429061\\
172.407963060357	0.0122873060537454\\
173.137833779374	0.0124357773537305\\
173.899730634716	0.0126183364541428\\
174.685161035718	0.0128324708346214\\
175.462621438348	0.0130668434981475\\
176.233710249371	0.0133179580562625\\
177.000171825469	0.0135828151560506\\
177.763435573061	0.0138586817399217\\
178.524704886545	0.0141430384789841\\
179.285021260482	0.0144335442671063\\
180.045308517652	0.0147280102660365\\
180.806404462281	0.0150243803694005\\
181.569084544967	0.0153207162101573\\
182.3274867653	0.0156126649121807\\
183.071460646514	0.0158945654197045\\
183.804026887023	0.0161661620130198\\
184.527678256927	0.0164271902072354\\
185.24431378944	0.0166773072851042\\
185.955435417096	0.0169161396083181\\
186.662270856609	0.0171433099397404\\
187.365851776875	0.0173584527953287\\
188.067066040969	0.0175612232544756\\
188.766693977737	0.017751301981574\\
189.465434585442	0.0179283980237439\\
190.163925090703	0.0180922502325745\\
190.862755869987	0.0182426277792793\\
191.562482367892	0.0183793301262042\\
192.263634759186	0.0185021865923196\\
192.966726013436	0.0186110556386425\\
193.6722591036	0.0187058239878782\\
194.38073322957	0.0187864055417535\\
195.092649944537	0.0188527402102937\\
195.80851889852	0.0189047925833878\\
196.528863729987	0.0189425504851268\\
197.25422824027	0.0189660233814125\\
197.985183112037	0.0189752406141703\\
198.72233333552	0.0189702494204202\\
199.466326874119	0.0189511126829711\\
200.217864775779	0.0189179063280543\\
200.977713359995	0.0188707162642622\\
201.747724343934	0.0188095463832123\\
202.52897379725	0.0187344277712964\\
203.322605764132	0.0186453903679499\\
204.129934829566	0.0185424516581347\\
204.95249119814	0.0184256065796041\\
205.792085107313	0.0182948133672656\\
206.650898457863	0.0181499733029816\\
207.531619990962	0.017990900653742\\
208.437653558656	0.0178172760202109\\
209.373456076334	0.0176285700154706\\
210.345123628299	0.0174239097638653\\
211.361501211961	0.0172018239416577\\
212.436559567364	0.0169596921516202\\
213.595521211185	0.0166923146158304\\
214.773708615059	0.0164158285335464\\
215.936234479576	0.0161403687038361\\
217.0825546228	0.015867935890886\\
218.218391188015	0.0155987595405196\\
219.34814126642	0.0153331635706058\\
220.472354085465	0.0150722037304589\\
221.560919348349	0.0148236984054293\\
222.624929397885	0.0145855956414708\\
223.67246833532	0.0143564795117821\\
224.708946366815	0.0141355024362899\\
225.738303979045	0.0139221126170311\\
226.763615863341	0.0137159205351646\\
227.787414925199	0.0135166295367579\\
228.811884708465	0.0133239962449223\\
229.838983344675	0.0131378063839218\\
230.870530033531	0.0129578590687478\\
231.908269738158	0.0127839561073145\\
232.953926036136	0.0126158942380478\\
234.009248041875	0.0124534591471311\\
235.076056017969	0.0122964204692884\\
236.156289872289	0.0121445271714291\\
237.252064674781	0.0119975028507753\\
238.365738524414	0.0118550404398408\\
239.500000813181	0.0117167956281579\\
240.65799199316	0.0115823781786691\\
241.843474408922	0.0114513396623114\\
243.061088181209	0.0113231551178129\\
244.316754857839	0.0111971940966376\\
245.61835780307	0.011072671699904\\
246.976985017293	0.0109485588829524\\
248.409468659076	0.010823398548318\\
249.944476955836	0.0106948629630834\\
251.641595050238	0.0105583621510639\\
253.357610742345	0.0104251666025664\\
255.028634373048	0.0102990812868692\\
256.67343657809	0.0101775790763531\\
258.290558338346	0.0100599153237615\\
259.850436823022	0.00994753493297418\\
261.372502603101	0.00983850192976854\\
262.868706170972	0.00973159030595749\\
264.350868429596	0.00962569491929455\\
265.827331379503	0.00952003641097132\\
267.304562725912	0.00941402644376665\\
268.788023633886	0.0093071943796128\\
270.282661101264	0.00919914519594911\\
271.793233206649	0.00908953190557749\\
273.324563665081	0.00897803472864678\\
274.881777385033	0.00886434293311944\\
276.470557954096	0.00874813630724502\\
278.097473269055	0.00862906307596201\\
279.770430170909	0.00850671031719563\\
281.499377124838	0.00838055922782888\\
283.29748534225	0.00824991058173608\\
285.18334078427	0.00811374635132434\\
287.185569746753	0.00797043600723479\\
289.354599478096	0.00781698321040036\\
291.803978887809	0.00764635109726208\\
294.263800014221	0.00747821951355787\\
296.66839230737	0.00731730272338915\\
299.035421199937	0.00716243996160482\\
301.296072393802	0.00701794493886754\\
303.509810823309	0.00687973734621294\\
305.711158675424	0.00674554292130292\\
307.922226984074	0.00661398074936204\\
310.160827507969	0.00648401170239421\\
312.443878544101	0.00635471875316163\\
314.789643153763	0.00622517879794393\\
317.220103439154	0.00609434501108475\\
319.764607744235	0.0059608820790269\\
322.467152727765	0.00582284782677367\\
325.405025527868	0.00567688004703071\\
328.758020652587	0.00551515033367844\\
332.359608637169	0.00534681184837027\\
335.806746726032	0.00519057167260528\\
338.970094475205	0.00505115675538714\\
342.018673729022	0.00492022636312861\\
345.044507864388	0.00479347933514118\\
348.094542406307	0.00466885775573733\\
351.204163159182	0.00454496848032393\\
354.407253993914	0.00442063038183516\\
357.741833806038	0.00429465469067847\\
361.256197956122	0.00416564702261679\\
365.019226547411	0.00403171005786729\\
369.143518605252	0.00388979744994955\\
373.856881659547	0.00373370751274138\\
378.637321020323	0.00358182704521548\\
383.062677641711	0.00344679621750199\\
387.429691027822	0.00331859656807232\\
391.878330359897	0.00319295947408915\\
396.500356685545	0.00306751093285373\\
401.389255205443	0.00294021729768179\\
406.66043311064	0.00280890627524\\
412.049581648717	0.00268072553051912\\
417.228072131693	0.00256306581563559\\
422.347392309865	0.00245181820296603\\
427.503506701113	0.00234464068904894\\
432.755895330962	0.00224026828636148\\
438.145378874319	0.00213799246064539\\
443.696909866936	0.00203751741030977\\
449.417123346912	0.0019389255662734\\
455.291823307406	0.00184263609305202\\
461.288881611296	0.00174927440343875\\
467.369255747492	0.00165944681795593\\
473.501154326418	0.00157353121719439\\
479.668439238791	0.00149160753586763\\
485.870099918996	0.00141352839972941\\
492.114699306059	0.00133903786172207\\
498.414532111917	0.00126786533947176\\
499.999999999996	0.00125055733624402\\
500	0.00125055733624391\\
500.000000000004	0.00125055733646462\\
500.31834636312	0.0206904117159519\\
500.63197508892	0.0400792232799314\\
500.94186526661	0.0594424725284449\\
501.248022346767	0.0787491805913189\\
501.550429943247	0.0979694348156854\\
501.849060964534	0.117074926848996\\
502.143877431601	0.136038810209727\\
502.434829998634	0.15483555213995\\
502.721857248483	0.173440782775464\\
503.004884783965	0.191831141973904\\
503.283824027904	0.209984117431035\\
503.558570695393	0.227877871217626\\
503.829002857205	0.245491049248963\\
504.094978462369	0.262802565169167\\
504.356332173736	0.279791349555035\\
504.61633241286	0.296660359389032\\
504.876756037587	0.313515186834752\\
505.137624040384	0.330347401412317\\
505.398961280752	0.347149013177475\\
505.660792649272	0.363912216573701\\
505.923143100671	0.380629385847399\\
506.186037720992	0.397293072867519\\
506.4495017252	0.413896000718962\\
506.713560522834	0.430431061814298\\
506.978239739937	0.446891313386126\\
507.2435652605	0.463269974352814\\
507.509563267939	0.479560422288676\\
507.776260274683	0.495756189756034\\
508.043683163986	0.511850961469022\\
508.311859241729	0.527838572095048\\
508.58081626006	0.543713002393692\\
508.850582471344	0.559468377194329\\
509.121186673108	0.575098962809857\\
509.392658252601	0.590599164402022\\
509.665027235827	0.605963523559363\\
509.938324351135	0.621186716615025\\
510.212581058258	0.636263550937371\\
510.487829632323	0.651188964182794\\
510.76410321284	0.665958021439067\\
511.041435875279	0.680565913447069\\
511.319862695145	0.695007954223103\\
511.599419820154	0.709279578907069\\
511.880144564626	0.723376342489064\\
512.162075479906	0.737293917017622\\
512.447281520475	0.751125607784497\\
512.735769205064	0.76486024676384\\
513.027473806076	0.778483344119175\\
513.322339582465	0.791981075719507\\
513.62031868974	0.805340224111959\\
513.921370725437	0.81854815039967\\
514.225462377318	0.831592770588906\\
514.532158048293	0.844445594709338\\
514.839113665298	0.85700291853856\\
515.146377641458	0.869264961383155\\
515.453994969585	0.881231911471041\\
515.762009997539	0.892904039213783\\
516.070466537511	0.904281697478305\\
516.379407987427	0.915365322299331\\
516.688877428121	0.9261554326515\\
516.998917728687	0.936652630493736\\
517.309571639308	0.946857600325507\\
517.620881890773	0.956771108916037\\
517.932891287949	0.966394004750365\\
518.24564279878	0.975727217260993\\
518.559179643238	0.984771755987008\\
518.873545381452	0.993528709627922\\
519.188784016366	1.0019992453945\\
519.504940069238	1.0101846074936\\
519.822058692538	1.01808611650819\\
520.140185739851	1.02570516753317\\
520.459367894441	1.0330432296506\\
520.779652753844	1.04010184414494\\
521.101088938615	1.04688262316837\\
521.42372621195	1.05338724844773\\
521.747615593556	1.05961746966802\\
522.072809487037	1.06557510293516\\
522.399361817185	1.07126202919416\\
522.727328170532	1.07668019246267\\
523.056765951483	1.08183159807961\\
523.387734542337	1.08671831074233\\
523.720295496728	1.09134245274907\\
524.054512729388	1.09570620182602\\
524.390452726196	1.09981178889947\\
524.728184779565	1.10366149581069\\
525.0677812576	1.10725765298288\\
525.409317882281	1.11060263669713\\
525.752874042345	1.1136988662383\\
526.098533146299	1.11654880086314\\
526.446383023346	1.11915493654489\\
526.796516352735	1.12151980224847\\
527.149031160491	1.1236459559581\\
527.504031382224	1.12553598027686\\
527.861627502721	1.12719247751075\\
528.221937271471	1.12861806408489\\
528.585086541503	1.12981536432788\\
528.951210221038	1.1307870033258\\
529.320453387644	1.13153559876711\\
529.692972573291	1.13206375147654\\
530.068937286825	1.13237403441541\\
530.448531796279	1.13246897971976\\
530.831957252729	1.13235106334551\\
531.219434243012	1.13202268668138\\
531.611205873162	1.13148615430046\\
532.007541507803	1.13074364681104\\
532.408741408665	1.12979718728787\\
532.81514251378	1.12864859926031\\
533.22712570976	1.12729945352194\\
533.645125173164	1.12575099964223\\
534.069569410878	1.12400438313191\\
534.489876333892	1.12211235329308\\
534.906481395403	1.12008769257058\\
535.319886938251	1.11794141957269\\
535.730519730943	1.11568363169861\\
536.138746216773	1.11332366005897\\
536.544885121084	1.11087018205814\\
536.949216906997	1.10833130954113\\
537.351990925577	1.10571465950073\\
537.75343096531	1.10302741136277\\
538.153739577841	1.10027635433931\\
538.553101583193	1.09746792643789\\
538.951686817448	1.09460824782775\\
539.349652484142	1.09170314839081\\
539.747144999831	1.08875819198371\\
540.144301603725	1.08577869684581\\
540.54125166842	1.08276975365174\\
540.938117836329	1.07973624112241\\
541.335017012362	1.07668283960964\\
541.732061191602	1.07361404334735\\
542.129358183106	1.07053417133217\\
542.527012311582	1.06744737654184\\
542.925124956556	1.06435765486501\\
543.32379509932	1.06126885266059\\
543.723119771484	1.05818467396424\\
544.123194518928	1.05510868662551\\
544.524113818759	1.05204432800317\\
544.925971433406	1.04899491045663\\
545.328860779638	1.04596362613697\\
545.732875300376	1.0429535512699\\
546.138108785078	1.03996765041762\\
546.544655711612	1.03700878023873\\
546.952611568274	1.03407969311972\\
547.362073222318	1.03118304025723\\
547.77313925954	1.02832137479976\\
548.185910288947	1.02549715511488\\
548.600489392474	1.02271274699263\\
549.016982445513	1.01997042673811\\
549.435498582041	1.01727238332099\\
549.856150614358	1.01462072093137\\
550.279055524849	1.01201746120848\\
550.704334998969	1.00946454541861\\
551.132115979452	1.00696383675944\\
551.562531319621	1.00451712239001\\
551.995720502807	1.00212611548632\\
552.431830390644	0.999792457600662\\
552.871016186603	0.9975177204085\\
553.313442403006	0.995303408138032\\
553.759284031189	0.993150959729681\\
554.20872788133	0.991061751110205\\
554.661974127868	0.989037097610377\\
555.11923808575	0.987078256644003\\
555.580752334552	0.985186430424409\\
556.046769204917	0.983362769046442\\
556.515720620883	0.981615070537262\\
556.985921847559	0.979950204264469\\
557.457558013575	0.978367527081591\\
557.930810389902	0.976866390268359\\
558.405865933807	0.975446110142143\\
558.882918221224	0.974105971696293\\
559.362168504754	0.972845232028454\\
559.843826923369	0.97166312362062\\
560.328113814574	0.970558857715897\\
560.816201270621	0.969529725269462\\
561.308419421652	0.968575221054261\\
561.80502285831	0.967695022759636\\
562.306111398754	0.966889041394221\\
562.810660702875	0.966158616602976\\
563.318904975146	0.965502748796326\\
563.831079935302	0.964920427370555\\
564.347436710021	0.964410613065376\\
564.868243968695	0.963972240900344\\
565.393790536509	0.963604223319292\\
565.924388475383	0.963305453786927\\
566.460376750467	0.963074810956647\\
567.002125628407	0.962911163569331\\
567.55004201467	0.962813376288037\\
568.104575977818	0.962780316753654\\
568.666228869286	0.962810864239264\\
569.23556349188	0.962903920451731\\
569.813217086472	0.963058423243627\\
570.399918155674	0.963273364363404\\
570.996508621679	0.963547812894327\\
571.603973678681	0.963880946931312\\
572.223482949026	0.964272097496361\\
572.856448749962	0.96472081116628\\
573.501644763855	0.965224525603344\\
574.137024006327	0.965761734107066\\
574.764145773858	0.966327729726457\\
575.384642535434	0.966918725568818\\
575.999803617571	0.967531315734958\\
576.610673107679	0.968162398297514\\
577.218119073226	0.968809125731627\\
577.822879665671	0.969468867708951\\
578.425594932892	0.970139182346469\\
579.026829571595	0.970817793601614\\
579.627089711959	0.971502573228428\\
580.226835503505	0.972191526014189\\
580.826490929648	0.972882777720733\\
581.426451567414	0.973574565036222\\
582.027090795178	0.974265227039075\\
582.628764962777	0.974953197989932\\
583.231817776277	0.975637001167766\\
583.836584091547	0.976315243539048\\
584.443393211499	0.97698661103764\\
585.05257200654	0.977649864554162\\
585.664447744137	0.978303836295083\\
586.279350818137	0.97894742655378\\
586.897617543546	0.979579600919125\\
587.519592897876	0.980199387663305\\
588.145633393036	0.980805875387828\\
588.776110262748	0.98139821098885\\
589.411412809974	0.981975597669186\\
590.051952218428	0.982537293171164\\
590.698165897278	0.983082608146615\\
591.350522480647	0.983610904613756\\
592.009527757211	0.984121594546332\\
592.675731690004	0.984614138494027\\
593.346232600269	0.985085643882741\\
594.020492179409	0.985535269291878\\
594.698958247129	0.985962955402503\\
595.382085260243	0.98636868018022\\
596.070346342781	0.98675246311557\\
596.764238116862	0.987114363997533\\
597.464286337627	0.987454481814959\\
598.171052250718	0.987772953603586\\
598.885140068764	0.988069953304827\\
599.607205769055	0.988345690562135\\
600.337967872062	0.988600409528022\\
601.078220612266	0.988834387569411\\
601.828850541268	0.989047933906849\\
602.589460674147	0.989241054240908\\
603.359008211947	0.989413779269412\\
604.138396061584	0.989566720666493\\
604.928595769641	0.989700505811649\\
605.730701267599	0.989815784841519\\
606.54595871224	0.989913229294146\\
607.375806945696	0.989993530614855\\
608.221932519015	0.990057398226809\\
609.086346893288	0.990105556933968\\
609.971602157534	0.990138746203225\\
610.8822727509	0.990157723443169\\
611.822377435019	0.99016315883839\\
612.783563594335	0.990155901551953\\
613.723953986774	0.99013820156005\\
614.64836061398	0.990112349692535\\
615.561265581139	0.990080188906207\\
616.465946272445	0.990043266471988\\
617.364923989596	0.990002918261525\\
618.260233270125	0.989960319548802\\
619.153581971464	0.989916518216269\\
620.04645296726	0.989872457716524\\
620.940173069121	0.989828993648641\\
621.835961724106	0.989786906206065\\
622.734967811695	0.989746909806749\\
623.638298735958	0.989709660787419\\
624.54704473215	0.989675763741352\\
625.462300486181	0.989645776902126\\
626.385185861108	0.989620216866845\\
627.316866671018	0.989599562912263\\
628.258577160292	0.989584261097919\\
629.211645342538	0.989574728362915\\
630.177523279443	0.989571356825071\\
631.157824100252	0.989574518545957\\
632.154369910289	0.989584571098\\
633.166557146571	0.989601809096247\\
634.188586209626	0.989626301033958\\
635.222210878138	0.989658146212235\\
636.269216077767	0.989697425963419\\
637.331565128841	0.989744210207753\\
638.411460454255	0.989798563691155\\
639.511425197274	0.98986055394547\\
640.634414975019	0.989930261901051\\
641.783977687813	0.990007796767288\\
642.964492910317	0.990093318048111\\
644.181548988249	0.990187070029461\\
645.442576059524	0.990289439623056\\
646.753128425805	0.990400640648068\\
648.108273023931	0.99051988012412\\
649.518140419641	0.990647552048967\\
650.98031922093	0.990782824147408\\
652.382354563349	0.990914334449859\\
653.744229773779	0.991042966846685\\
655.08197945345	0.99116949450775\\
656.405460711129	0.991294257266406\\
657.721501941295	0.991417401820033\\
659.035336253604	0.991538988547353\\
660.352065436261	0.991659109885913\\
661.676191626664	0.991777833227592\\
663.011271289277	0.991895163632675\\
664.360804306684	0.992011120165971\\
665.728380089987	0.992125743482878\\
667.117823492944	0.992239103129324\\
668.533358306492	0.9923513055033\\
669.973195665075	0.992462003047902\\
671.427428221109	0.992570313159364\\
672.900513952957	0.9926765100479\\
674.396978433219	0.992780872070536\\
675.92180633806	0.99288370673777\\
677.480697068384	0.992985362995956\\
679.080419437341	0.993086247843074\\
680.72934165776	0.993186850910942\\
682.438290277085	0.993287784368366\\
684.22206928707	0.993389853922683\\
686.102454814173	0.993494199821894\\
688.115060365621	0.993602622503399\\
690.306383309738	0.99371735297712\\
692.380404783726	0.993823214293818\\
694.386099532054	0.993923439715568\\
696.362557481768	0.994020435714536\\
698.331632341184	0.99411556514539\\
700.308461075455	0.994209754256262\\
702.305490986239	0.994303723433176\\
704.33432121914	0.994398090994133\\
706.406897045228	0.994493435913814\\
708.536603663039	0.994590349029997\\
710.711818677589	0.994688245809659\\
712.931029872237	0.994786983079148\\
715.212172479621	0.99488724282694\\
717.576530492177	0.994989784273856\\
720.052823738271	0.995095601142674\\
722.684603899091	0.9952061932403\\
725.54967341424	0.995324279634117\\
728.679176480554	0.995450377878894\\
731.84893541316	0.995574902640198\\
734.849396567784	0.995689734939111\\
737.797198748619	0.995799634290071\\
740.758705311609	0.995907129865387\\
743.722646392544	0.996011815142281\\
746.689735142594	0.996113757945714\\
749.698372424421	0.996214284545485\\
752.782376478684	0.996314448010108\\
755.977543485294	0.996415254431148\\
759.327796180672	0.996517838876821\\
762.894604162248	0.996623700534655\\
766.777799928225	0.996735200315778\\
771.16198933914	0.996856611940023\\
775.327885081216	0.996967791906527\\
779.440383971789	0.997073723569697\\
783.618456608122	0.997177609122558\\
787.943843994626	0.997281337098548\\
792.38178141451	0.997383865467217\\
797.002855303264	0.997486571968845\\
801.903615768175	0.997591134565775\\
807.052899524138	0.99769634206938\\
812.237984740189	0.997797646817247\\
817.485745144042	0.997895636614227\\
822.814351756779	0.997990664777409\\
828.168458256792	0.998081813001198\\
833.644743211414	0.998170753947958\\
839.250685464292	0.998257522642439\\
844.940491573278	0.998341377786523\\
850.705810034808	0.998422230106005\\
856.538391422323	0.998500017852541\\
862.432760405386	0.998574736996035\\
868.386827239582	0.998646437721358\\
874.401459831418	0.998715208444537\\
880.479487435946	0.998781156579877\\
886.624604783139	0.998844392536625\\
892.840474861193	0.998905019557506\\
899.130159511185	0.998963129327519\\
905.495891690834	0.999018801994331\\
911.939132489805	0.999072108848301\\
918.460814444516	0.999123115994923\\
925.061657477152	0.999171887719513\\
931.742454944057	0.999218488764783\\
938.504263857806	0.999262985307648\\
945.348478431896	0.999305444855072\\
952.276805103903	0.999345935502344\\
959.291179949474	0.999384525013197\\
966.393672376379	0.999421280050567\\
973.586408458017	0.999456265711454\\
980.871530672681	0.999489545368929\\
988.251193935047	0.999521180729445\\
995.727587821901	0.999551231990867\\
1000	0.99956755027536\\
};
\addlegendentry{$\text{y}_\text{2}$};

\end{axis}
\end{tikzpicture}%
}
  \caption{Step responses in outputs $y_1$ (\texttt{blue}) and $y_2$
    (\texttt{red}).}
  \label{fig:minimum_phase_step_23}
\end{figure}

Tables \ref{tbl:minimum_phase_overshoot} and \ref{tbl:minimum_phase_risetime}
feature the overshoot and rise time performance metrics with regard to the
step responses of the system. It is evident that all inputs are coupled to all
outputs, but the applied control actions result in swift and satisfactory
performance.

Figure \ref{fig:minimum_phase_bode_l_21} shows the frequency response of
$L_{11}(j\omega) = G_{11}(j\omega) f_1(j\omega)$ (\texttt{blue}) and
$L_{22} = G_{22}(j\omega) f_2(j\omega)$ (\texttt{red}), where $L = GF$. The
phase margin is $60^{\circ}$ at the required gain crossover frequency
$\omega_c^{mp} = 0.1$ rad/s.

\noindent\makebox[\linewidth][c]$ & $8.6\%$  \\
          $u_2$          & $9.1\%$  & $\underline{13.2\%}$ \\
        \end{tabular}
        \caption{Overshoot figures regarding the step responses depicted in
          figure \ref{fig:minimum_phase_step_23}. Underlined percentages
          reference control-coupled inputs/outputs.}
        \label{tbl:minimum_phase_overshoot}
    \end{table}
  \end{minipage}
  \hfill
  \begin{minipage}{0.45\linewidth}
    \begin{table}[H]\centering
        \begin{tabular}{r|c|c}
        Input/Output & $y_1$    & $y_2$    \\ \hline
          $u_1$          & $14.2$ & ~  \\
          $u_2$          & ~  & $14.3$ \\
        \end{tabular}
        \caption{Rise times in seconds regarding the step responses depicted in
          figure \ref{fig:minimum_phase_step_23}.}
        \label{tbl:minimum_phase_risetime}
    \end{table}
  \end{minipage}
\end{minipage}
}


\begin{figure}[H]\centering
  \scalebox{1}{% This file was created by matlab2tikz.
%
%The latest updates can be retrieved from
%  http://www.mathworks.com/matlabcentral/fileexchange/22022-matlab2tikz-matlab2tikz
%where you can also make suggestions and rate matlab2tikz.
%
\definecolor{mycolor1}{rgb}{0.00000,0.44700,0.74100}%
\definecolor{mycolor2}{rgb}{0.85000,0.32500,0.09800}%
%
\begin{tikzpicture}

\begin{axis}[%
width=4.008in,
height=1.551in,
at={(0.818in,1.941in)},
scale only axis,
unbounded coords=jump,
separate axis lines,
every outer x axis line/.append style={white!40!black},
every x tick label/.append style={font=\color{white!40!black}},
xmode=log,
xmin=0.001,
xmax=10,
xtick={0.001,0.01,0.1,1,10},
xticklabels={\empty},
xminorticks=true,
every outer y axis line/.append style={white!40!black},
every y tick label/.append style={font=\color{white!40!black}},
ymin=-50,
ymax=50,
ylabel={Magnitude [dB]},
axis background/.style={fill=white}
]
\addplot [color=mycolor1,solid,forget plot]
  table[row sep=crcr]{%
1e-20	384.869736177478\\
1.12894362780737e-18	343.816291045683\\
1.12894362780737e-13	243.816291045683\\
1.12894362780737e-09	163.816291045683\\
1.12894362780736e-06	103.816291044138\\
0.000112894362780737	63.8162756031642\\
0.00112894362780736	43.8147471334721\\
0.00131669607745723	42.4779253656916\\
0.00153567327693626	41.1409034049739\\
0.00179106815450566	39.803609248255\\
0.00208893726436671	38.4659450546304\\
0.00243634441463467	37.1277779240023\\
0.00284152818180544	35.788927427423\\
0.00331409728423198	34.4491487811412\\
0.00386525844778892	33.1081102092887\\
0.00450808216743882	31.7653626110819\\
0.00525781266709484	30.4202991425034\\
0.00613222940831371	29.0721017734365\\
0.00715206872080618	27.7196713804804\\
0.00834151555350901	26.36153768196\\
0.00972877700783467	24.9957457006989\\
0.0113467512541359	23.6197171388282\\
0.0132338076943846	22.2300891888375\\
0.0154346968722098	20.8225415242559\\
0.0180016117083285	19.3916365027765\\
0.0209954252280068	17.9307195135908\\
0.0244871341326	16.4319549640798\\
0.0285595424487085	14.8866015613855\\
0.0333092251817945	13.2856409471473\\
0.0388488185412673	11.6208385707924\\
0.0453096910485086	9.88620859768464\\
0.0528450588717525	8.07967914062047\\
0.0616336192663337	6.20457243516578\\
0.0718837882854156	4.27043668038816\\
0.0838386432562625	2.29287469577659\\
0.0977816872302615	0.292280324956268\\
0.114043572107573	-1.70831404586406\\
0.133009939871749	-3.68587603047563\\
0.15513056788504	-5.62001178525324\\
0.180930035120228	-7.49511849070793\\
0.21102016227302	-9.3016479477721\\
0.246114520765675	-11.0362779208798\\
0.2870453547152	-12.7010802972347\\
0.334783317161619	-14.302040911473\\
0.390460488590524	-15.8473943141672\\
0.455397223621958	-17.3461588636782\\
0.531133462520644	-18.807075852864\\
0.619465250063431	-20.2379808743434\\
0.722487327789547	-21.645528538925\\
0.842642809686306	-23.0351564889157\\
0.982781119342844	-24.4111850507864\\
1.14622556252078	-25.7769770320475\\
1.33685213758949	-27.1351107305678\\
1.55918145277378	-28.4875411235239\\
1.81848592998268	-29.8357384925908\\
2.12091483750145	-31.1808019611693\\
2.47364011663078	-32.5235495593762\\
2.88502646047473	-33.8645881312287\\
3.36482967820566	-35.2043667775105\\
3.92442804890965	-36.5432172740897\\
4.57709215144635	-37.8813844047178\\
5.33829956919514	-39.2190485983424\\
6.22610193274434	-40.5563427550613\\
7.26155300474592	-41.8933647157791\\
8.46920795874155	-43.2301864835596\\
84.6920795874154	-63.2317149532516\\
8469.20795874154	-103.231730394226\\
8469207.95874155	-163.23173039577\\
84692079587.4155	-243.23173039577\\
8.46920795874155e+15	-343.23173039577\\
1e+20	-424.674874456528\\
};
\addplot [color=black,solid,forget plot]
  table[row sep=crcr]{%
nan	0\\
nan	-inf\\
};
\addplot [color=black,dotted,forget plot]
  table[row sep=crcr]{%
nan	-50\\
nan	0\\
};
\addplot [color=black,dotted,forget plot]
  table[row sep=crcr]{%
0.100000001215841	-50\\
0.100000001215841	0\\
};
\addplot [color=black,dotted,forget plot]
  table[row sep=crcr]{%
0.001	0\\
10	0\\
};
\addplot [color=mycolor2,solid,forget plot]
  table[row sep=crcr]{%
1e-20	385.248702973641\\
1.03744402507685e-18	344.9294094974\\
1.03744402507685e-13	244.9294094974\\
1.03744402507685e-09	164.9294094974\\
1.03744402507685e-06	104.929409495854\\
0.000103744402507685	64.9293940349411\\
0.00103744402507685	44.9278635912338\\
0.00121008773024377	43.5902627602193\\
0.00141146151454103	42.2524612062936\\
0.00164634642368378	40.9143866868666\\
0.00192031912939387	39.5759410179631\\
0.00223988433155204	38.2369908096145\\
0.00261262919372985	36.8973549354753\\
0.00304740347873223	35.556787621942\\
0.00355452966095484	34.2149556926294\\
0.0041460480040746	32.8714080720227\\
0.00483600242274342	31.5255351426627\\
0.00564077391525527	30.1765149973762\\
0.00657946948359345	28.8232431239731\\
0.0076743757746544	27.4642418050527\\
0.00895148821306418	26.0975458993525\\
0.010441128188336	24.7205633865515\\
0.0121786629497161	23.3299132361482\\
0.014205345300566	21.9212514545223\\
0.0165692930284288	20.4891105767187\\
0.0193266313245478	19.0267999405173\\
0.0225428253163324	17.5264428261672\\
0.0262942343499463	15.9792547839348\\
0.0306699249250265	14.3761776427573\\
0.035773785324488	12.708947762794\\
0.0417269921452685	10.9715687503437\\
0.0486708872907404	9.16198219074471\\
0.0567703337307663	7.2835483438108\\
0.0662176297023401	5.34587284681092\\
0.0772370742823366	3.36462697706207\\
0.0900902927288616	1.36027661393634\\
0.105082448026234	-0.644073749189376\\
0.122569486109001	-2.62531961893822\\
0.142966586782163	-4.56299511593811\\
0.166758020980564	-6.44142896287202\\
0.194508648399961	-8.25101552247106\\
0.226877328478186	-9.98839453492135\\
0.264632563131876	-11.6556244148846\\
0.308670742640906	-13.2587015560621\\
0.360037427876963	-14.8058895982945\\
0.419952174162036	-16.3062467126446\\
0.489837486128497	-17.7685573488461\\
0.571352591031248	-19.2006982266496\\
0.666432832362866	-20.6093600082755\\
0.777335619060669	-22.0000101586788\\
0.906694021238474	-23.3769926714798\\
1.05757928492073	-24.74368857718\\
1.23357375001292	-26.1026898961004\\
1.43885590273736	-27.4559617695035\\
1.6782995818616	-28.80498191479\\
1.95758969408834	-30.15085484415\\
2.28335718593827	-31.4944024647567\\
2.66333647664814	-32.8362343940693\\
3.10654908987869	-34.1768017076026\\
3.62351784404337	-35.5164375817418\\
4.22651668659561	-36.8553877900904\\
4.92986210387691	-38.1938334589939\\
5.75025302522057	-39.5319079784209\\
6.70716729136406	-40.8697095323466\\
7.82332410018048	-42.2073103633611\\
78.2332410018048	-62.2088408070684\\
7823.32410018048	-102.208856267981\\
7823324.10018048	-162.208856269527\\
78233241001.8049	-242.208856269527\\
7.82332410018049e+15	-342.208856269527\\
1e+20	-424.341029823103\\
};
\addplot [color=black,solid,forget plot]
  table[row sep=crcr]{%
nan	0\\
nan	-inf\\
};
\addplot [color=black,dotted,forget plot]
  table[row sep=crcr]{%
nan	-50\\
nan	0\\
};
\addplot [color=black,dotted,forget plot]
  table[row sep=crcr]{%
0.100000005653169	-50\\
0.100000005653169	0\\
};
\addplot [color=black,dotted,forget plot]
  table[row sep=crcr]{%
0.001	0\\
10	0\\
};
\end{axis}

\begin{axis}[%
width=4.008in,
height=1.376in,
at={(0.818in,0.44in)},
scale only axis,
unbounded coords=jump,
separate axis lines,
every outer x axis line/.append style={white!40!black},
every x tick label/.append style={font=\color{white!40!black}},
xmode=log,
xmin=0.001,
xmax=10,
xlabel={Frequency [rad/s]},
xminorticks=true,
every outer y axis line/.append style={white!40!black},
every y tick label/.append style={font=\color{white!40!black}},
ymin=-130.4,
ymax=-89.6,
ytick={-130, -120, -110, -100,  -90},
ylabel={Phase [deg]},
axis background/.style={fill=white}
]
\addplot [color=mycolor1,solid,forget plot]
  table[row sep=crcr]{%
1e-20	-90\\
1.12894362780737e-18	-90\\
1.12894362780737e-13	-90.0000000000764\\
1.12894362780737e-09	-90.0000007640398\\
1.12894362780736e-06	-90.0007640398262\\
0.000112894362780737	-90.0764038354999\\
0.00112894362780736	-90.7638927286726\\
0.00131669607745723	-90.8908724751106\\
0.00153567327693626	-91.0389337615612\\
0.00179106815450566	-91.2115614005552\\
0.00208893726436671	-91.4128073404662\\
0.00243634441463467	-91.6473777344041\\
0.00284152818180544	-91.9207302739103\\
0.00331409728423198	-92.23918106466\\
0.00386525844778892	-92.6100188246605\\
0.00450808216743882	-93.0416216878212\\
0.00525781266709484	-93.5435678519402\\
0.00613222940831371	-94.1267249303581\\
0.00715206872080618	-94.8032930583401\\
0.00834151555350901	-95.5867621672562\\
0.00972877700783467	-96.4917228137202\\
0.0113467512541359	-97.5334413507824\\
0.0132338076943846	-98.7270745347627\\
0.0154346968722098	-100.086360662031\\
0.0180016117083285	-101.621597666086\\
0.0209954252280068	-103.336731947967\\
0.0244871341326	-105.22548346461\\
0.0285595424487085	-107.266685082186\\
0.0333092251817945	-109.419461276661\\
0.0388488185412673	-111.619466602905\\
0.0453096910485086	-113.777923560164\\
0.0528450588717525	-115.785233881985\\
0.0616336192663337	-117.520103384049\\
0.0718837882854156	-118.863458413157\\
0.0838386432562625	-119.714625341594\\
0.0977816872302615	-120.006242434347\\
0.114043572107573	-119.714625341594\\
0.133009939871749	-118.863458413157\\
0.15513056788504	-117.520103384049\\
0.180930035120228	-115.785233881985\\
0.21102016227302	-113.777923560164\\
0.246114520765675	-111.619466602905\\
0.2870453547152	-109.419461276661\\
0.334783317161619	-107.266685082186\\
0.390460488590524	-105.22548346461\\
0.455397223621958	-103.336731947967\\
0.531133462520644	-101.621597666086\\
0.619465250063431	-100.086360662031\\
0.722487327789547	-98.7270745347627\\
0.842642809686306	-97.5334413507824\\
0.982781119342844	-96.4917228137202\\
1.14622556252078	-95.5867621672562\\
1.33685213758949	-94.8032930583401\\
1.55918145277378	-94.1267249303581\\
1.81848592998268	-93.5435678519402\\
2.12091483750145	-93.0416216878212\\
2.47364011663078	-92.6100188246605\\
2.88502646047473	-92.2391810646599\\
3.36482967820566	-91.9207302739103\\
3.92442804890965	-91.6473777344041\\
4.57709215144635	-91.4128073404662\\
5.33829956919514	-91.2115614005552\\
6.22610193274434	-91.0389337615612\\
7.26155300474592	-90.8908724751106\\
8.46920795874155	-90.7638927286726\\
84.6920795874154	-90.0764038354999\\
8469.20795874154	-90.0007640398262\\
8469207.95874155	-90.0000007640398\\
84692079587.4155	-90.0000000000764\\
8.46920795874155e+15	-90\\
1e+20	-90\\
};
\addplot [color=black,solid,forget plot]
  table[row sep=crcr]{%
0.100000001215841	-180\\
0.100000001215841	-119.999999993234\\
};
\addplot [color=black,dotted,forget plot]
  table[row sep=crcr]{%
0.100000001215841	-180\\
0.100000001215841	-89.6\\
};
\addplot [color=black,dotted,forget plot]
  table[row sep=crcr]{%
nan	-180\\
nan	-89.6\\
};
\addplot [color=black,dotted,forget plot]
  table[row sep=crcr]{%
0.001	-180\\
10	-180\\
};
\addplot [color=mycolor2,solid,forget plot]
  table[row sep=crcr]{%
1e-20	-90\\
1.03744402507685e-18	-90\\
1.03744402507685e-13	-90.0000000000766\\
1.03744402507685e-09	-90.0000007660185\\
1.03744402507685e-06	-90.0007660184814\\
0.000103744402507685	-90.0766017009373\\
0.00103744402507685	-90.7658712964827\\
0.00121008773024377	-90.893260019145\\
0.00141146151454103	-91.0418115301651\\
0.00164634642368378	-91.2150263053663\\
0.00192031912939387	-91.4169748829974\\
0.00223988433155204	-91.6523854478945\\
0.00261262919372985	-91.9267417542003\\
0.00304740347873223	-92.2463906632699\\
0.00355452966095484	-92.618657067647\\
0.0041460480040746	-93.0519614583154\\
0.00483600242274342	-93.5559313213092\\
0.00564077391525527	-94.1414911252324\\
0.00657946948359345	-94.8209057822177\\
0.0076743757746544	-95.6077377190762\\
0.00895148821306418	-96.5166565137085\\
0.010441128188336	-97.5630112432744\\
0.0121786629497161	-98.7620397523643\\
0.014205345300566	-100.127550836064\\
0.0165692930284288	-101.669888648485\\
0.0193266313245478	-103.393002470569\\
0.0225428253163324	-105.290548146285\\
0.0262942343499463	-107.341202723791\\
0.0306699249250265	-109.503824740641\\
0.035773785324488	-111.713691150584\\
0.0417269921452685	-113.881560267281\\
0.0486708872907404	-115.897336272039\\
0.0567703337307663	-117.63926250189\\
0.0662176297023401	-118.987896892485\\
0.0772370742823366	-119.842318940575\\
0.0900902927288616	-120.135034510782\\
0.105082448026234	-119.842318940575\\
0.122569486109001	-118.987896892485\\
0.142966586782163	-117.63926250189\\
0.166758020980564	-115.897336272039\\
0.194508648399961	-113.881560267281\\
0.226877328478186	-111.713691150584\\
0.264632563131876	-109.503824740641\\
0.308670742640906	-107.341202723791\\
0.360037427876963	-105.290548146285\\
0.419952174162036	-103.393002470569\\
0.489837486128497	-101.669888648485\\
0.571352591031248	-100.127550836064\\
0.666432832362866	-98.7620397523644\\
0.777335619060669	-97.5630112432744\\
0.906694021238474	-96.5166565137085\\
1.05757928492073	-95.6077377190762\\
1.23357375001292	-94.8209057822177\\
1.43885590273736	-94.1414911252324\\
1.6782995818616	-93.5559313213092\\
1.95758969408834	-93.0519614583154\\
2.28335718593827	-92.618657067647\\
2.66333647664814	-92.2463906632699\\
3.10654908987869	-91.9267417542003\\
3.62351784404337	-91.6523854478945\\
4.22651668659561	-91.4169748829974\\
4.92986210387691	-91.2150263053663\\
5.75025302522057	-91.0418115301651\\
6.70716729136406	-90.893260019145\\
7.82332410018048	-90.7658712964827\\
78.2332410018048	-90.0766017009373\\
7823.32410018048	-90.0007660184814\\
7823324.10018048	-90.0000007660185\\
78233241001.8049	-90.0000000000766\\
7.82332410018049e+15	-90\\
1e+20	-90\\
};
\addplot [color=black,solid,forget plot]
  table[row sep=crcr]{%
0.100000005653169	-180\\
0.100000005653169	-119.999999854163\\
};
\addplot [color=black,dotted,forget plot]
  table[row sep=crcr]{%
0.100000005653169	-180\\
0.100000005653169	-89.6\\
};
\addplot [color=black,dotted,forget plot]
  table[row sep=crcr]{%
nan	-180\\
nan	-89.6\\
};
\addplot [color=black,dotted,forget plot]
  table[row sep=crcr]{%
0.001	-180\\
10	-180\\
};
\end{axis}
\end{tikzpicture}%
}
  \caption{The frequency response of $L_{11}(j\omega) = G_{11}(j\omega) f_1(j\omega)$
    (\texttt{blue}) and $L_{22} = G_{22}(j\omega) f_2(j\omega)$ (\texttt{red}),
    where $L = GF$. The phase margin is $60^{\circ}$ and the gain crossover
    frequency is $\omega_c^{mp} = 0.1$ rad/s.}
  \label{fig:minimum_phase_bode_l_21}
\end{figure}
