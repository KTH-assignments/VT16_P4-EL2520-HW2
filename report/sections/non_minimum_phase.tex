For the non-minimum phase case, the Relative Gain Array suggests that inputs
$u_1$ and $u_2$ should be paired to outputs $y_2$ and $y_1$ respectively. The
controller $F(s)$ will thus be of the form:

\[
F(s)=
\begin{bmatrix}
  0      & f_1(s) \\
  f_2(s) & 0      \\
\end{bmatrix}
\]
where $f_i = K_i (1 + \dfrac{1}{T_i s}), i \in \{1,2\}$. Here,
$$T_i = \dfrac{1}{\omega_c^{nmp}} tan(\phi_m - \pi/2 - arg(G_{ji}(j\omega_c^{nmp})))$$
and
$$K_i = \dfrac{1}{\Big|G_{ji}(j\omega_c^{nmp}) (1 + \dfrac{1}{j\omega_c^{nmp}T_i})\Big|}$$
where $i \neq j$, $\omega_c^{nmp} = 0.02$ rad/s and $\phi_m = 60^{\circ}$. From
the above two relations we conclude that

\begin{align*}
  f_1(s) &= 0.1437 (1 + \dfrac{1}{4.8107s})  \\
  f_2(s) &= 0.1469 (1 + \dfrac{1}{3.9426s})  \\
\end{align*}

Figure \ref{fig:non_minimum_phase_step_23} shows the step responses of the
system for an input step signal applied at input $u_1$ at $t=100$s and at input
$u_2$ at $t=500$s.

\begin{figure}[H]\centering
  \scalebox{0.8}{% This file was created by matlab2tikz.
%
%The latest updates can be retrieved from
%  http://www.mathworks.com/matlabcentral/fileexchange/22022-matlab2tikz-matlab2tikz
%where you can also make suggestions and rate matlab2tikz.
%
\definecolor{mycolor1}{rgb}{0.00000,0.44700,0.74100}%
\definecolor{mycolor2}{rgb}{0.85000,0.32500,0.09800}%
%
\begin{tikzpicture}

\begin{axis}[%
width=4.133in,
height=3.26in,
at={(0.693in,0.44in)},
scale only axis,
xmin=0,
xmax=1000,
xlabel={Time [sec]},
xmajorgrids,
ymin=-0.2,
ymax=1.4,
ylabel={Amplitude},
ymajorgrids,
axis background/.style={fill=white},
title style={font=\bfseries}
]
\addplot [color=mycolor1,solid]
  table[row sep=crcr]{%
0	0\\
20	0\\
40	0\\
60	0\\
80	0\\
99.9999999999991	0\\
100	0\\
100.000000000001	1.53808952828952e-28\\
100.250000000001	1.1728126348202e-05\\
100.500000000001	4.73289144599184e-05\\
100.850544231243	0.00013859636352824\\
101.202517950558	0.000280219216585856\\
101.556888941341	0.00047489011911646\\
101.913679628515	0.000725090399973189\\
102.272922004152	0.00103327861534731\\
102.634648570296	0.00140188722841944\\
102.998892441743	0.00183332237467616\\
103.365687356436	0.00232996356010088\\
103.735067692587	0.00289416334545049\\
104.107068482125	0.00352824700477283\\
104.481725425198	0.00423451216171324\\
104.85907490935	0.00501522841128261\\
105.239154020458	0.00587263690142173\\
105.62200056231	0.00680894990494813\\
106.007653073665	0.00782635036120774\\
106.396150843809	0.0089269913855434\\
106.787533932726	0.0101129957619189\\
107.181843189012	0.0113864554005187\\
107.579120266997	0.0127494307605592\\
107.979407648463	0.0142039502555106\\
108.382748659856	0.0157520096093723\\
108.789187493935	0.0173955711927994\\
109.198769229919	0.0191365633174794\\
109.61153986032	0.0209768795218355\\
110.027546306976	0.0229183777723276\\
110.44683644619	0.0249628796613797\\
110.869459131682	0.0271121695544312\\
111.295464219712	0.0293679937040268\\
111.724902594199	0.0317320593195498\\
112.157826189645	0.0342060335773721\\
112.594288027069	0.0367915426572718\\
113.034342226357	0.0394901705933811\\
113.478044054148	0.0423034582834896\\
113.925449930164	0.0452329021929536\\
114.376617476271	0.0482799532681179\\
114.831605542833	0.0514460156560076\\
115.29047423048	0.0547324453262322\\
115.753284944498	0.0581405488635614\\
116.220100410635	0.0616715819283322\\
116.690984730959	0.0653267479456011\\
117.166003399395	0.069107196427948\\
117.645223341086	0.0730140214023521\\
118.128712996345	0.077048260146114\\
118.616542304868	0.0812108910417879\\
119.108782793931	0.0855028322113898\\
119.605507606072	0.0899249395826889\\
120.106791546322	0.0944780050399727\\
120.612711150858	0.0991627546983229\\
121.123344729042	0.103979846866127\\
121.639046606272	0.108932526237294\\
122.161663289837	0.114038829599914\\
122.69126131225	0.119300008593528\\
123.227933007518	0.124717422850069\\
123.771772715104	0.130292306653854\\
124.322877173504	0.136025766103024\\
124.881345541205	0.14191877252719\\
125.44727949037	0.147972156559949\\
126.020783245639	0.154186601551849\\
126.601963688648	0.160562637604681\\
127.19093044259	0.167100635336867\\
127.78779596515	0.173800799683514\\
128.39267565483	0.180663163799889\\
129.005687978283	0.187687583170238\\
129.626954633387	0.194873730112481\\
130.256600671701	0.202221087810353\\
130.894754712045	0.20972894540091\\
131.541549132981	0.217396392841465\\
132.197120327155	0.225222316530703\\
132.861608945794	0.233205394847834\\
133.535160213818	0.241344094568628\\
134.217924278033	0.249636667720592\\
134.910056628503	0.258081149360579\\
135.611718538761	0.266675355606396\\
136.323077611539	0.275416882959684\\
137.044308343278	0.284303107834229\\
137.77559289882	0.293331188603876\\
138.51712183387	0.302498066887671\\
139.269095104306	0.311800472176014\\
140.031723066345	0.321234926049294\\
140.805227790498	0.330797749852305\\
141.589844438701	0.340485072563715\\
142.385823000318	0.350292842309517\\
143.193430278029	0.360216839903839\\
144.012952119457	0.370252694040152\\
144.844696168653	0.380395901088737\\
145.68899507737	0.39064184821873\\
146.546210293348	0.400985840613111\\
147.416768592614	0.411423516301626\\
148.301173070788	0.421950936146695\\
149.199907264498	0.432563403847236\\
150.113510075432	0.443256329957405\\
151.042584241262	0.454025281690233\\
151.987808732533	0.46486606621154\\
152.949954333505	0.47577483404419\\
153.929903307921	0.486748207402386\\
154.928675000426	0.497783446841284\\
155.942945165956	0.508829872929345\\
156.970528576739	0.51985085848543\\
158.011888499813	0.530838921081683\\
159.067503744285	0.541786423255539\\
160.137894029229	0.552685822504219\\
161.223624941086	0.563529691048205\\
162.325313990777	0.574310739722443\\
163.424624930627	0.584841720803799\\
164.518154794317	0.595089456177611\\
165.606849846856	0.605063610000913\\
166.691576452987	0.614772913999021\\
167.773108615537	0.624225093493407\\
168.852145666648	0.633427072085283\\
169.92932628582	0.642385130932539\\
171.00523962646	0.651105032872749\\
172.080434324649	0.65959212063125\\
173.155425875389	0.667851395123484\\
174.230702809314	0.675887578846818\\
175.308860740902	0.683720412543034\\
176.396179093134	0.691393425016674\\
177.493059984596	0.69890562739833\\
178.599923567877	0.706256176656739\\
179.717203055488	0.713444336353416\\
180.845347529136	0.720469492603312\\
181.984824887498	0.727331168790681\\
183.136125035685	0.734029039551268\\
184.299763369731	0.740562944126958\\
185.47628463236	0.746932899279367\\
186.666267249516	0.753139112047324\\
187.85020862116	0.759082942780222\\
189.027231034637	0.764769181237611\\
190.1985338403	0.770212696630856\\
191.365162152316	0.775426764125037\\
192.528025519617	0.780423317476465\\
193.687926248941	0.785213210924394\\
194.845580825777	0.78980641737824\\
196.001636852145	0.794212182527988\\
197.156684992512	0.798439140041144\\
198.311270451362	0.802495410221867\\
199.465900402503	0.806388671983332\\
200.621050824374	0.810126224026429\\
201.777172500171	0.813715035746557\\
202.934695182438	0.817161786328763\\
204.094031942549	0.820472899775428\\
205.255582656643	0.823654573278158\\
206.419736915778	0.826712800725865\\
207.586876654382	0.829653392797643\\
208.757378982596	0.832481995228368\\
209.931617708582	0.835204101882665\\
211.109966382878	0.837825069874647\\
212.292799553501	0.840350128977446\\
213.480495032874	0.842784392224373\\
214.673435930687	0.845132864807975\\
215.872012153894	0.847400450999753\\
217.076622843881	0.849591962104418\\
218.28767823284	0.851712122480138\\
219.505601631486	0.853765575205927\\
220.730831819963	0.855756887866799\\
221.96382575784	0.857690558264516\\
223.205060827724	0.859571018883643\\
224.455038089642	0.861402642525549\\
225.714285673779	0.863189747655483\\
226.983362384984	0.864936603660359\\
228.262862314041	0.866647437138439\\
229.553419215694	0.868326437508365\\
230.855712531445	0.869977764507728\\
232.170474045736	0.87160555613872\\
233.498495519473	0.873213937554495\\
234.840638144178	0.87480703191458\\
236.19784398858	0.876388973355047\\
237.571148530169	0.877963921045526\\
238.961698376129	0.879536079115803\\
240.370771061531	0.881109718794401\\
241.799801429748	0.88268920797548\\
243.250413803611	0.884279047342078\\
244.724465097653	0.885883918884856\\
246.22409999884	0.88750874822906\\
247.751826654975	0.889158790335128\\
249.310620442462	0.890839747349688\\
250.904069288792	0.892557934203502\\
252.536588638538	0.894320524013986\\
254.213744809118	0.896135918202818\\
255.942773248902	0.898014340498117\\
257.733454898977	0.899968842682759\\
259.577129195323	0.901992278816365\\
261.425212276555	0.904032842583154\\
263.279126746184	0.906092753961733\\
265.140249585578	0.908173796081648\\
267.009922647692	0.910277349347528\\
268.889470887232	0.912404430194168\\
270.780218839522	0.914555724684851\\
272.683506133929	0.916731618221869\\
274.568201262146	0.918894853080801\\
276.423466871146	0.921030870221066\\
278.255453714919	0.923144560296628\\
280.069163381585	0.925239541584941\\
281.868538738725	0.927318236475086\\
283.656790650766	0.929382230551167\\
285.436607957873	0.931432504298236\\
287.21029862347	0.933469589453464\\
288.979885283109	0.935493675050459\\
290.747175159536	0.937504685186744\\
292.513810682433	0.939502335238488\\
294.281308907151	0.941486175457272\\
296.051092322491	0.943455624673466\\
297.824512982502	0.945409996183505\\
299.602873197929	0.947348519355807\\
301.387443323877	0.949270357438714\\
303.179476993943	0.951174621888878\\
304.980224907204	0.953060384364994\\
306.790948916637	0.954926688116381\\
308.61293475285	0.956772556862964\\
310.447505309249	0.958597003057413\\
312.29603503243	0.96039903583048\\
314.159965028516	0.962177667951575\\
316.040819740857	0.963931922347248\\
317.940226744103	0.965660839195268\\
319.85993860093	0.967363482042309\\
321.801860328685	0.969038945484766\\
323.768082564213	0.970686363388346\\
325.760920938133	0.97230491688937\\
327.782967625358	0.973893845495244\\
329.837157394773	0.975452460565835\\
331.926852778311	0.976980161496284\\
334.055960774983	0.978476459155364\\
336.229090909543	0.979941006438302\\
338.451782753768	0.981373644779977\\
340.730834324758	0.982774470439031\\
343.074809632724	0.98414394280866\\
345.494851035002	0.98548306215657\\
348.006070665293	0.986793685613688\\
350.630145049542	0.988079140715278\\
353.400739583501	0.989345560598852\\
356.222619697081	0.990542507595283\\
359.063214637405	0.991656070981138\\
361.92739235034	0.992689825014062\\
364.818768462255	0.993646710960371\\
367.716721278192	0.994522616002692\\
370.535502634753	0.995298874273463\\
373.273442276453	0.995985297260812\\
375.957271545531	0.99659720143206\\
378.604778801614	0.997145188419332\\
381.228265664024	0.997637042344327\\
383.836966452296	0.998078854583254\\
386.438269266593	0.998475607031215\\
389.038397964798	0.998831505817504\\
391.642832506505	0.99915018786919\\
394.256591550095	0.999434856372922\\
396.884434089921	0.999688373108322\\
399.531020659154	0.999913324159519\\
402.201057684257	1.00011206812106\\
404.899424181033	1.00028677116041\\
407.631317684836	1.00043943451193\\
410.402415043238	1.00057191517551\\
413.216519599716	1.00068584737918\\
416.067625012628	1.00078248215577\\
418.961420939457	1.000863513644\\
421.903976647934	1.0009305057567\\
424.902136407342	1.00098491074358\\
427.963775427084	1.00102807665131\\
431.098154729865	1.00106125284207\\
434.316427258766	1.00108559355787\\
437.632400908053	1.00110215942463\\
441.063743901609	1.00111191624888\\
444.634014013364	1.00111572970687\\
448.376382570209	1.00111435258368\\
452.341258748374	1.0011083957589\\
456.614734088221	1.00109825562938\\
460.709802682674	1.0010860439078\\
464.575359821415	1.00107301922833\\
468.31125019419	1.00105951565298\\
471.971229670403	1.0010457034905\\
475.585702305993	1.00103166929535\\
479.174946365461	1.00101743524948\\
482.754073505136	1.00100297505108\\
486.33527673573	1.00098822605536\\
489.929025421511	1.0009730984317\\
493.544796856815	1.00095748211183\\
497.191564993164	1.00094125219962\\
500	1.00092837416212\\
500.000000000004	1.00092837416212\\
500.368660538377	1.00206477477979\\
500.73732107675	1.0032638058562\\
501.113526615474	1.00455058143967\\
501.492453131862	1.00590981282322\\
501.874190135464	1.0073417932668\\
502.258777285174	1.00884663552403\\
502.646255598411	1.01042444233095\\
503.036666893361	1.01207530358045\\
503.430053860042	1.01379929571825\\
503.826460050093	1.01559648077771\\
504.225929919622	1.01746690560399\\
504.628508845003	1.01941060092871\\
505.034243149703	1.02142758045715\\
505.443180112135	1.0235178398217\\
505.855367997198	1.02568135560826\\
506.270856086805	1.02791808434183\\
506.689694724804	1.03022796151407\\
507.11193529961	1.03261090022542\\
507.537630293437	1.03506679013602\\
507.966833327581	1.03759549636253\\
508.399599162604	1.0401968580642\\
508.83598374077	1.04287068721894\\
509.276044229556	1.04561676736262\\
509.71983903275	1.04843485207497\\
510.167427808283	1.05132466343976\\
510.618871548301	1.05428589085009\\
511.074232586738	1.05731818929638\\
511.533574626962	1.06042117771301\\
511.996962803599	1.06359443749327\\
512.464463683985	1.06683751053159\\
512.936145346208	1.07014989772107\\
513.412077385464	1.07353105689017\\
513.892330984234	1.07698040111325\\
514.376978950903	1.08049729673232\\
514.8660957498	1.08408106123893\\
515.359757537808	1.08773096112337\\
515.858042262817	1.09144621010624\\
516.361029665823	1.09522596658238\\
516.868801354638	1.09906933151114\\
517.381440861815	1.10297534611084\\
517.899033680387	1.10694298929969\\
518.421667349142	1.11097117542855\\
518.949431517612	1.11505875177586\\
519.482417965202	1.11920449559685\\
520.02072075714	1.12340711214462\\
520.564436230313	1.12766523128023\\
521.113663143722	1.13197740527145\\
521.668502708556	1.13634210555825\\
522.229058713907	1.14075772017177\\
522.795437600136	1.1452225506457\\
523.367748577077	1.14973480918139\\
523.94610372657	1.15429261558849\\
524.53061811947	1.15889399423603\\
525.12140994543	1.16353687099764\\
525.718600649908	1.16821907014368\\
526.322315093505	1.1729383112944\\
526.932681718086	1.17769220630581\\
527.549832715642	1.18247825605172\\
528.173904223059	1.18729384727547\\
528.806628964095	1.19214842974859\\
529.448139596249	1.19703881351452\\
530.098575722896	1.20196165377695\\
530.758081030682	1.20691342341045\\
531.42680360118	1.21189040998702\\
532.104896210629	1.21688871266903\\
532.792516643393	1.22190423915949\\
533.489828103167	1.22693270330525\\
534.196999682667	1.23196962301512\\
534.914206826747	1.23701031799232\\
535.641632022252	1.24204990887919\\
536.379465391873	1.24708331545038\\
537.127905570897	1.25210525642102\\
537.887160619088	1.25711024899769\\
538.657449088967	1.26209260887264\\
539.439001323047	1.26704645092818\\
540.232060879826	1.27196568975882\\
541.036886280998	1.2768440409586\\
541.853752982252	1.2816750221845\\
542.682955764174	1.28645195476052\\
543.524811535877	1.29116796518553\\
544.379662580995	1.29581598606667\\
545.247880478545	1.30038875696679\\
546.129870801907	1.30487882463251\\
547.026112036236	1.309278702427\\
547.92697804712	1.31353370174171\\
548.829534832106	1.31762616933147\\
549.734075778223	1.32155461242919\\
550.640876340126	1.32531747637258\\
551.550204151415	1.32891319456634\\
552.462320964399	1.33234019810799\\
553.377484362099	1.33559692348962\\
554.295949411115	1.33868181910936\\
555.217970103879	1.34159335010824\\
556.14380077326	1.34433000227063\\
557.073697455472	1.34689028490467\\
558.007919247132	1.34927273284648\\
558.946729545328	1.35147590732204\\
559.890397468122	1.35349839643045\\
560.839199155688	1.35533881444505\\
561.793419240539	1.35699580059264\\
562.753352335181	1.3584680168859\\
563.719304655542	1.35975414517792\\
564.691595770016	1.36085288330011\\
565.670560552665	1.36176294025305\\
566.656551290831	1.36248303023439\\
567.649940064685	1.36301186546313\\
568.651121469713	1.36334814759072\\
569.660515654213	1.36349055740755\\
570.674663439099	1.36343831791456\\
571.678743858934	1.36319735726744\\
572.672728812591	1.36277704503629\\
573.656678759666	1.36218656359453\\
574.630598720889	1.36143493782954\\
575.606026499585	1.36051924832452\\
576.583256556652	1.35944228504609\\
577.562536750855	1.35820683959052\\
578.544109020351	1.35681568485325\\
579.528210673409	1.35527158056337\\
580.515075657844	1.35357727771531\\
581.504935736865	1.35173552212211\\
582.4980215652	1.34974905726132\\
583.494563723886	1.34762062644429\\
584.494793690333	1.34535297446282\\
585.498944797005	1.3429488487014\\
586.507253153069	1.34041099985123\\
587.519958577837	1.33774218218675\\
588.537305537984	1.33494515347959\\
589.559544103479	1.33202267456779\\
590.586930925946	1.32897750862246\\
591.619730282459	1.32581242003199\\
592.658215172364	1.32253017297337\\
593.702668474309	1.31913352969355\\
594.753384208847	1.31562524839689\\
595.810668901625	1.31200808078779\\
596.87484308252	1.30828476919165\\
597.946242920872	1.30445804329641\\
599.02522208007	1.30053061626913\\
600.112153741693	1.29650518047425\\
601.207432938964	1.29238440235173\\
602.311479163784	1.28817091665009\\
603.424739346282	1.2838673197373\\
604.547691284358	1.27947616179705\\
605.680847602987	1.27499993773141\\
606.824760358316	1.2704410764846\\
607.980026440224	1.26580192838735\\
609.147293932439	1.26108475015091\\
610.327269704551	1.25629168674835\\
611.520728499458	1.25142474955571\\
612.728523965994	1.24648578950265\\
613.951602123841	1.24147646400766\\
615.191018048854	1.23639819549682\\
616.447956745448	1.23125211893216\\
617.723759628398	1.22603901442982\\
619.009427977459	1.22080203007773\\
620.303077349176	1.21555508619967\\
621.605253729647	1.21030214261503\\
622.91650712123	1.20504713593108\\
624.237415961649	1.19979389282989\\
625.568591101678	1.19454612796595\\
626.91068057269	1.18930744019448\\
628.264374890928	1.18408130851927\\
629.630413326398	1.17887108652291\\
631.009591182555	1.17367999566611\\
632.402768433986	1.16851111680973\\
633.810879776757	1.16336738058532\\
635.234946801763	1.15825155500203\\
636.676092927799	1.15316622936665\\
638.135561402226	1.14811379517464\\
639.614737886034	1.14309642088615\\
641.115178634594	1.13811602019461\\
642.638646705473	1.13317420968942\\
644.187158507528	1.12827225398635\\
645.763045810368	1.12341098977265\\
647.369038644352	1.11859072325485\\
649.008380450494	1.11381108304109\\
650.684990882901	1.10907080846858\\
652.380943276601	1.10442868720275\\
654.092345836784	1.09990019908444\\
655.810436136311	1.09551121384596\\
657.536694000575	1.09125920567935\\
659.272547557538	1.08714188615461\\
661.019410273931	1.08315710382176\\
662.778701179284	1.07930279608406\\
664.551863772882	1.07557695101326\\
666.338736965334	1.07198081978714\\
668.128711522152	1.06853485149817\\
669.923726592657	1.0652329357543\\
671.725615294753	1.06206950875136\\
673.528675148101	1.05905163362032\\
675.310698896807	1.05620998301145\\
677.075714232289	1.05352953382949\\
678.827325148277	1.05099723864034\\
680.568492320662	1.04860208250831\\
682.301708943003	1.04633456789633\\
684.029123483751	1.0441863577121\\
685.752624928984	1.04215002503968\\
687.473904867979	1.04021887128393\\
689.194503684603	1.03838679130408\\
690.915845529427	1.03664817160967\\
692.639266647147	1.0349978109214\\
694.366037274631	1.03343085887647\\
696.097380722517	1.03194276639095\\
697.834489328548	1.03052924604015\\
699.578538029954	1.0291862401293\\
701.330698304893	1.02790989316841\\
703.092149293825	1.02669653045707\\
704.8640904991	1.02554263875334\\
706.647753326961	1.02444485067222\\
708.444414039504	1.02339993068295\\
710.255406697758	1.02240476343212\\
712.082138615965	1.02145634275771\\
713.926107186685	1.02055176209975\\
715.788919190887	1.01968820568004\\
717.672314421463	1.01886293972358\\
719.578193131302	1.01807330421145\\
721.508650871281	1.01731670390939\\
723.466020855369	1.01659059915829\\
725.452928540107	1.01589249519128\\
727.47236243318	1.01521992947087\\
729.527767407267	1.01457045610248\\
731.62317209414	1.01394162526704\\
733.763365207312	1.01333095559061\\
735.954149923374	1.01273589452035\\
738.20272122583	1.01215375982125\\
740.518255785909	1.01158164737638\\
742.912880639819	1.01101627862316\\
745.403384857543	1.01045372768354\\
748.014541500365	1.00988888479021\\
750.701114914338	1.00933158851493\\
753.409816233175	1.00879166275216\\
756.130757943517	1.00826930228138\\
758.857773315495	1.00776405205941\\
761.58572298362	1.00727542813582\\
764.319675705989	1.0068013791673\\
767.064284544192	1.00634029634982\\
769.82394111321	1.00589093398977\\
772.602900688429	1.00545234886488\\
775.321885948516	1.00503642224224\\
777.994947730275	1.00464006621214\\
780.636911380246	1.00426048177097\\
783.258501769735	1.00389579355989\\
785.867935595949	1.00354471412653\\
788.471869901863	1.00320634537657\\
791.075954066315	1.00288006091825\\
793.685174967769	1.00256543162049\\
796.304091660541	1.00226217546525\\
798.937003706805	1.00197012256913\\
801.588082597459	1.00168918969887\\
804.261481362966	1.00141936128159\\
806.961434318712	1.00116067486367\\
809.692353066343	1.00091320999444\\
812.458930364115	1.00067707931294\\
815.26625690306	1.00045242160165\\
818.119960291442	1.00023939645319\\
821.026381421749	1.00003818015666\\
823.992807363255	0.999848962747373\\
827.027787458028	0.999671946454925\\
830.141587155872	0.999507345350448\\
833.346862749555	0.999355386763021\\
836.659718130802	0.999216315032201\\
840.086478819769	0.999090907257833\\
843.52984283198	0.998982882439207\\
846.997158112128	0.998891336668438\\
850.491649274277	0.998815501457312\\
853.994062255339	0.998754924033719\\
857.511904146047	0.998708470907589\\
861.052161566385	0.998675093460431\\
864.621583000596	0.998653806750701\\
868.226866238895	0.99864367579288\\
871.8748220102	0.998643806060626\\
875.572528409004	0.998653336933197\\
879.327488239979	0.998671437259442\\
883.101543229174	0.998696949234996\\
886.879487242804	0.998728663996369\\
890.673132358855	0.998765623810658\\
894.49321853986	0.998806981413736\\
898.349972653889	0.998851978774152\\
902.253545997893	0.998899932892633\\
906.214391548349	0.998950226053881\\
910.243631699019	0.999002299099461\\
914.353457620403	0.999055646865368\\
918.557612274302	0.999109815397029\\
922.872016769796	0.999164400727843\\
927.315639001475	0.999219049344946\\
931.911770982395	0.99927346082727\\
936.690024012066	0.999327393824277\\
941.689657355228	0.999380677900716\\
946.958517132953	0.999433169112663\\
952.27602026242	0.999482269270584\\
957.666095489574	0.999528107774007\\
962.931799918129	0.999569190312997\\
968.12404470104	0.999606298289292\\
973.288891647574	0.999640070139378\\
978.459952924227	0.999670976426917\\
983.664715747529	0.99969939184034\\
988.927935334399	0.999725630347475\\
994.273717213273	0.999749963594055\\
1000	0.999773712910636\\
};
\addlegendentry{$\text{y}_\text{1}$};

\addplot [color=mycolor2,solid]
  table[row sep=crcr]{%
0	0\\
20	0\\
40	0\\
60	0\\
80	0\\
99.9999999999991	0\\
100	0\\
100.000000000001	2.4152560351908e-15\\
100.250000000001	0.000680983075245594\\
100.500000000001	0.00139584116684168\\
100.850544231243	0.00245442133012451\\
101.202517950558	0.00358227365213102\\
101.556888941341	0.00478235380969535\\
101.913679628515	0.00605483490584401\\
102.272922004152	0.00739991013434846\\
102.634648570296	0.0088177638939162\\
102.998892441743	0.0103085714885255\\
103.365687356436	0.0118724984967478\\
103.735067692587	0.0135097001438943\\
104.107068482125	0.0152203206349332\\
104.481725425198	0.0170044924655877\\
104.85907490935	0.0188623357295919\\
105.239154020458	0.020793957359353\\
105.62200056231	0.0227994503792611\\
106.007653073665	0.0248788931186708\\
106.396150843809	0.0270323483861008\\
106.787533932726	0.0292598626367958\\
107.181843189012	0.0315614650964099\\
107.579120266997	0.0339371668456874\\
107.979407648463	0.0363869598973194\\
108.382748659856	0.0389108162100866\\
108.789187493935	0.0415086866923074\\
109.198769229919	0.044180500157884\\
109.61153986032	0.0469261622868196\\
110.027546306976	0.0497455544749298\\
110.44683644619	0.0526385326998965\\
110.869459131682	0.055604926330046\\
111.295464219712	0.0586445369034031\\
111.724902594199	0.0617571368611222\\
112.157826189645	0.0649424682171812\\
112.594288027069	0.0682002412750938\\
113.034342226357	0.0715301331205561\\
113.478044054148	0.074931786327354\\
113.925449930164	0.0784048072972019\\
114.376617476271	0.0819487648696491\\
114.831605542833	0.0855631887052444\\
115.29047423048	0.0892475675715747\\
115.753284944498	0.093001347831374\\
116.220100410635	0.0968239315631831\\
116.690984730959	0.100714674944039\\
117.166003399395	0.104672886240534\\
117.645223341086	0.108697823927248\\
118.128712996345	0.112788695119072\\
118.616542304868	0.116944653099221\\
119.108782793931	0.121164795653032\\
119.605507606072	0.125448162822988\\
120.106791546322	0.129793734756717\\
120.612711150858	0.134200429669477\\
121.123344729042	0.138667101503564\\
121.639046606272	0.143194949049325\\
122.161663289837	0.147798276281707\\
122.69126131225	0.152475703744079\\
123.227933007518	0.15722593129286\\
123.771772715104	0.162047524448534\\
124.322877173504	0.166938912727857\\
124.881345541205	0.171898384613838\\
125.44727949037	0.176924083095563\\
126.020783245639	0.182014000652496\\
126.601963688648	0.187165974760154\\
127.19093044259	0.192377683162573\\
127.78779596515	0.197646639169146\\
128.39267565483	0.202970187026792\\
129.005687978283	0.208345497439151\\
129.626954633387	0.21376956336376\\
130.256600671701	0.219239195421096\\
130.894754712045	0.224751018069364\\
131.541549132981	0.230301465558759\\
132.197120327155	0.23588677838007\\
132.861608945794	0.241502999578879\\
133.535160213818	0.247145971611133\\
134.217924278033	0.252811333391346\\
134.910056628503	0.258494517835848\\
135.611718538761	0.264190749406293\\
136.323077611539	0.269895042316764\\
137.044308343278	0.275602198628536\\
137.77559289882	0.281306807674368\\
138.51712183387	0.287003244609263\\
139.269095104306	0.292685670626801\\
140.031723066345	0.298348032355685\\
140.805227790498	0.3039840627056\\
141.589844438701	0.3095872810039\\
142.385823000318	0.3151509943016\\
143.193430278029	0.320668298612135\\
144.012952119457	0.326132079629117\\
144.844696168653	0.331535014227688\\
145.68899507737	0.336869571549557\\
146.546210293348	0.342128013560895\\
147.416768592614	0.347302582128065\\
148.301173070788	0.352385497374673\\
149.199907264498	0.357368357098805\\
150.113510075432	0.362242546844727\\
151.042584241262	0.366999223533668\\
151.987808732533	0.37162930460991\\
152.949954333505	0.376123452400321\\
153.929903307921	0.380472051334784\\
154.928675000426	0.38466517740642\\
155.942945165956	0.388675268135278\\
156.970528576739	0.392482211885449\\
158.011888499813	0.396077101107213\\
159.067503744285	0.399450861279478\\
160.137894029229	0.402594326948333\\
161.223624941086	0.405498228611659\\
162.325313990777	0.408153176414921\\
163.424624930627	0.410511397434642\\
164.518154794317	0.412571053413188\\
165.606849846856	0.414340683052861\\
166.691576452987	0.415828037444649\\
167.773108615537	0.417040169577034\\
168.852145666648	0.417983564846978\\
169.92932628582	0.41866424264431\\
171.00523962646	0.419087836105411\\
172.080434324649	0.419259655415809\\
173.155425875389	0.419184738469955\\
174.230702809314	0.418867891696195\\
175.308860740902	0.418312395110814\\
176.396179093134	0.417516431758041\\
177.493059984596	0.416479336454334\\
178.599923567877	0.415200513697335\\
179.717203055488	0.413679446908618\\
180.845347529136	0.411915702733774\\
181.984824887498	0.409908933568616\\
183.136125035685	0.407658878213109\\
184.299763369731	0.405165360559467\\
185.47628463236	0.402428286151218\\
186.666267249516	0.399447636331433\\
187.85020862116	0.396279006302445\\
189.027231034637	0.392936761687165\\
190.1985338403	0.389429177767478\\
191.365162152316	0.385764036225663\\
192.528025519617	0.381948747772412\\
193.687926248941	0.377990407494894\\
194.845580825777	0.37389583810536\\
196.001636852145	0.369671622566653\\
197.156684992512	0.365324135023598\\
198.311270451362	0.360859558598307\\
199.465900402503	0.356283906645805\\
200.621050824374	0.351603037546946\\
201.777172500171	0.346822665522133\\
202.934695182438	0.341948373158908\\
204.094031942549	0.336985618659576\\
205.255582656643	0.331939742907372\\
206.419736915778	0.326815975997097\\
207.586876654382	0.321619442568376\\
208.757378982596	0.316355164201771\\
209.931617708582	0.311028066043343\\
211.109966382878	0.305642975287164\\
212.292799553501	0.300204626679626\\
213.480495032874	0.294717662656321\\
214.673435930687	0.289186633912175\\
215.872012153894	0.283616002030488\\
217.076622843881	0.278010137524293\\
218.28767823284	0.272373320480147\\
219.505601631486	0.266709740673358\\
220.730831819963	0.261023496036669\\
221.96382575784	0.255318590050499\\
223.205060827724	0.249598931873392\\
224.455038089642	0.243868332552153\\
225.714285673779	0.238130501601944\\
226.983362384984	0.232389043851405\\
228.262862314041	0.226647453241637\\
229.553419215694	0.220909109547429\\
230.855712531445	0.215177269961979\\
232.170474045736	0.209455060696914\\
233.498495519473	0.203745467661804\\
234.840638144178	0.198051323369985\\
236.19784398858	0.192375290478137\\
237.571148530169	0.186719846994182\\
238.961698376129	0.181087257686802\\
240.370771061531	0.175479547095165\\
241.799801429748	0.16989845858187\\
243.250413803611	0.164345406859689\\
244.724465097653	0.158821409332704\\
246.22409999884	0.153327000013861\\
247.751826654975	0.147862105492945\\
249.310620442462	0.142425872372739\\
250.904069288792	0.137016423016274\\
252.536588638538	0.131630481560262\\
254.213744809118	0.126262804528728\\
255.942773248902	0.12090524327767\\
257.733454898977	0.11554512537648\\
259.577129195323	0.110226149181852\\
261.425212276555	0.105096621762935\\
263.279126746184	0.100152441043311\\
265.140249585578	0.0953896685710331\\
267.009922647692	0.0908044947535346\\
268.889470887232	0.0863931883537091\\
270.780218839522	0.082152057175477\\
272.683506133929	0.0780774171463985\\
274.568201262146	0.0742303014457233\\
276.423466871146	0.0706208897795793\\
278.255453714919	0.0672248313109812\\
280.069163381585	0.0640222731346648\\
281.868538738725	0.060997051564482\\
283.656790650766	0.0581356226971846\\
285.436607957873	0.055426370719981\\
287.21029862347	0.0528591414671203\\
288.979885283109	0.0504249209926024\\
290.747175159536	0.0481156024955177\\
292.513810682433	0.0459238160407625\\
294.281308907151	0.0438427990241368\\
296.051092322491	0.0418662968555602\\
297.824512982502	0.0399884861649636\\
299.602873197929	0.0382039131306479\\
301.387443323877	0.0365074439089494\\
303.179476993943	0.0348942248115517\\
304.980224907204	0.0333596496233719\\
306.790948916637	0.0318993315518363\\
308.61293475285	0.0305090805520954\\
310.447505309249	0.0291848837323432\\
312.29603503243	0.0279228881661425\\
314.159965028516	0.0267193861845427\\
316.040819740857	0.025570802353959\\
317.940226744103	0.0244736811851247\\
319.85993860093	0.0234246763420263\\
321.801860328685	0.0224205394670122\\
323.768082564213	0.0214581091164955\\
325.760920938133	0.0205342999500706\\
327.782967625358	0.0196460899311653\\
329.837157394773	0.0187905056928088\\
331.926852778311	0.0179646053866412\\
334.055960774983	0.017165455847729\\
336.229090909543	0.0163901031772429\\
338.451782753768	0.0156355303962925\\
340.730834324758	0.0148985974918371\\
343.074809632724	0.0141759474100798\\
345.494851035002	0.0134638548267141\\
348.006070665293	0.0127579638494431\\
350.630145049542	0.0120527915914652\\
353.400739583501	0.0113406743800937\\
356.222619697081	0.0106462552666227\\
359.063214637405	0.00997571974397449\\
361.92739235034	0.00932599733744643\\
364.818768462255	0.00869481435742325\\
367.716721278192	0.00808534253440052\\
370.535502634753	0.00751342516868214\\
373.273442276453	0.00697669862173611\\
375.957271545531	0.00646788916757557\\
378.604778801614	0.00598230490026719\\
381.228265664024	0.0055168369474321\\
383.836966452296	0.00506931933450994\\
386.438269266593	0.00463819794040199\\
389.038397964798	0.00422234149357203\\
391.642832506505	0.0038209242052536\\
394.256591550095	0.0034333481255111\\
396.884434089921	0.00305918990255316\\
399.531020659154	0.00269816217604713\\
402.201057684257	0.00235008429086236\\
404.899424181033	0.00201486132321582\\
407.631317684836	0.00169246615154695\\
410.402415043238	0.00138292568510257\\
413.216519599716	0.00108656989410871\\
416.067625012628	0.000804722597322893\\
418.961420939457	0.00053745583491116\\
421.903976647934	0.000284893831892896\\
424.902136407342	4.71855115244946e-05\\
427.963775427084	-0.000175499467578888\\
431.098154729865	-0.00038297318520164\\
434.316427258766	-0.000575029871618726\\
437.632400908053	-0.0007514441605323\\
441.063743901609	-0.000911965994102593\\
444.634014013364	-0.00105631038442389\\
448.376382570209	-0.00118413901884495\\
452.341258748374	-0.00129502603346099\\
456.614734088221	-0.00138839169856064\\
460.709802682674	-0.00145458084765449\\
464.575359821415	-0.00149811436848335\\
468.31125019419	-0.00152442230297356\\
471.971229670403	-0.00153672128876248\\
475.585702305993	-0.00153721658513994\\
479.174946365461	-0.00152759933903845\\
482.754073505136	-0.00150925018042658\\
486.33527673573	-0.00148334084521351\\
489.929025421511	-0.00145089186688829\\
493.544796856815	-0.00141280803065635\\
497.191564993164	-0.00136990146324312\\
500	-0.00133438269250297\\
500.000000000004	-0.00133438269250297\\
500.368660538377	-0.00129899541446976\\
500.73732107675	-0.00120149374357292\\
501.113526615474	-0.0010366429016635\\
501.492453131862	-0.000802628756875645\\
501.874190135464	-0.000496753485803803\\
502.258777285174	-0.000116365569256205\\
502.646255598411	0.000341158232872729\\
503.036666893361	0.000878412148379804\\
503.430053860042	0.00149796101706445\\
503.826460050093	0.00220233979032081\\
504.225929919622	0.00299405306974854\\
504.628508845003	0.00387557457442833\\
505.034243149703	0.00484934659104508\\
505.443180112135	0.00591777933885007\\
505.855367997198	0.00708325034362511\\
506.270856086805	0.00834810377546868\\
506.689694724804	0.00971464979840708\\
507.11193529961	0.0111851636757452\\
507.537630293437	0.0127618850236735\\
507.966833327581	0.0144470170279296\\
508.399599162604	0.0162427254393682\\
508.83598374077	0.0181511376663182\\
509.276044229556	0.0201743418324796\\
509.71983903275	0.0223143856334037\\
510.167427808283	0.0245732751389403\\
510.618871548301	0.0269529738472355\\
511.074232586738	0.0294554013287855\\
511.533574626962	0.0320824318824386\\
511.996962803599	0.0348358933163473\\
512.464463683985	0.0377175653094982\\
512.936145346208	0.0407291781456015\\
513.412077385464	0.0438724109376646\\
513.892330984234	0.0471488901679207\\
514.376978950903	0.0505601879543032\\
514.8660957498	0.0541078201596715\\
515.359757537808	0.0577932444413248\\
515.858042262817	0.0616178586760671\\
516.361029665823	0.0655829985825854\\
516.868801354638	0.069689935771539\\
517.381440861815	0.0739398755882434\\
517.899033680387	0.0783339546647653\\
518.421667349142	0.0828732387753951\\
518.949431517612	0.0875587204312875\\
519.482417965202	0.0923913159460956\\
520.02072075714	0.0973718636144962\\
520.564436230313	0.102501120250673\\
521.113663143722	0.107779759111025\\
521.668502708556	0.11320836658669\\
522.229058713907	0.118787439680729\\
522.795437600136	0.124517382867211\\
523.367748577077	0.130398505270466\\
523.94610372657	0.136431017571048\\
524.53061811947	0.142615028933904\\
525.12140994543	0.148950543946906\\
525.718600649908	0.15543745951219\\
526.322315093505	0.162075561853841\\
526.932681718086	0.168864523476767\\
527.549832715642	0.175803900023759\\
528.173904223059	0.18289312728365\\
528.806628964095	0.190149865380446\\
529.448139596249	0.197573639523587\\
530.098575722896	0.205163807364097\\
530.758081030682	0.212919518486511\\
531.42680360118	0.220839709832184\\
532.104896210629	0.228923101015415\\
532.792516643393	0.237168189817789\\
533.489828103167	0.245573248866519\\
534.196999682667	0.254136323039436\\
534.914206826747	0.262855226791911\\
535.641632022252	0.271727544214519\\
536.379465391873	0.280750627878067\\
537.127905570897	0.289921601024797\\
537.887160619088	0.299237360031694\\
538.657449088967	0.308694578531653\\
539.439001323047	0.318289713969459\\
540.232060879826	0.328019015208063\\
541.036886280998	0.337878533348121\\
541.853752982252	0.347864134306721\\
542.682955764174	0.357971515237467\\
543.524811535877	0.368196224276868\\
544.379662580995	0.378533683455128\\
545.247880478545	0.388979216933154\\
546.129870801907	0.399528084947666\\
547.026112036236	0.410175918257948\\
547.92697804712	0.420799136764878\\
548.829534832106	0.431355618203093\\
549.734075778223	0.441841883390117\\
550.640876340126	0.452254394707305\\
551.550204151415	0.462589663482193\\
552.462320964399	0.472844261404174\\
553.377484362099	0.483014829824599\\
554.295949411115	0.493098088669428\\
555.217970103879	0.503090843120712\\
556.14380077326	0.512989989990647\\
557.073697455472	0.522792523376523\\
558.007919247132	0.532495539960784\\
558.946729545328	0.542096242696224\\
559.890397468122	0.551591945860676\\
560.839199155688	0.560980078234746\\
561.793419240539	0.570258187238629\\
562.753352335181	0.579423942323063\\
563.719304655542	0.588475138634715\\
564.691595770016	0.597409700651286\\
565.670560552665	0.606225686294942\\
566.656551290831	0.614921290806585\\
567.649940064685	0.623494851174037\\
568.651121469713	0.631944851352893\\
569.660515654213	0.640269927612949\\
570.674663439099	0.648437777669984\\
571.678743858934	0.656330843025774\\
572.672728812591	0.663955235918851\\
573.656678759666	0.671317898483196\\
574.630598720889	0.678425447683535\\
575.606026499585	0.685365601970714\\
576.583256556652	0.692140902330713\\
577.562536750855	0.698753565487511\\
578.544109020351	0.705205786089665\\
579.528210673409	0.711499747017534\\
580.515075657844	0.717637628725363\\
581.504935736865	0.72362161718777\\
582.4980215652	0.729453910514175\\
583.494563723886	0.735136724672048\\
584.494793690333	0.740672298239973\\
585.498944797005	0.746062896554209\\
586.507253153069	0.751310815141453\\
587.519958577837	0.756418382738887\\
588.537305537984	0.761387963870737\\
589.559544103479	0.766221961073477\\
590.586930925946	0.770922816797416\\
591.619730282459	0.775493015184578\\
592.658215172364	0.779935083644561\\
593.702668474309	0.784251594252182\\
594.753384208847	0.788445165139624\\
595.810668901625	0.792518461822913\\
596.87484308252	0.796474198566502\\
597.946242920872	0.80031513973948\\
599.02522208007	0.804044101412312\\
600.112153741693	0.807663952939307\\
601.207432938964	0.8111776189377\\
602.311479163784	0.814588081427797\\
603.424739346282	0.81789838235327\\
604.547691284358	0.821111626578969\\
605.680847602987	0.824230985438478\\
606.824760358316	0.827259700971459\\
607.980026440224	0.830201091044801\\
609.147293932439	0.83305855550628\\
610.327269704551	0.835835583741588\\
611.520728499458	0.838535763874673\\
612.728523965994	0.841162794182225\\
613.951602123841	0.843720497212745\\
615.191018048854	0.846212837568208\\
616.447956745448	0.848643944396596\\
617.723759628398	0.851018140220026\\
619.009427977459	0.853321465892935\\
620.303077349176	0.855554585744639\\
621.605253729647	0.857722641534741\\
622.91650712123	0.859830656748491\\
624.237415961649	0.861883580782765\\
625.568591101678	0.863886295164063\\
626.91068057269	0.865843620425646\\
628.264374890928	0.867760323213923\\
629.630413326398	0.869641124243207\\
631.009591182555	0.871490707111974\\
632.402768433986	0.873313728419205\\
633.810879776757	0.875114829212077\\
635.234946801763	0.876898648655413\\
636.676092927799	0.878669840642317\\
638.135561402226	0.880433093652342\\
639.614737886034	0.88219315565026\\
641.115178634594	0.883954865138518\\
642.638646705473	0.885723191161203\\
644.187158507528	0.88750328487272\\
645.763045810368	0.889300548531408\\
647.369038644352	0.891120728098124\\
649.008380450494	0.892970042428505\\
650.684990882901	0.894855366677074\\
652.380943276601	0.896758966768864\\
654.092345836784	0.898678761373195\\
655.810436136311	0.900606916890506\\
657.536694000575	0.902546747643214\\
659.272547557538	0.904501175510268\\
661.019410273931	0.906472788465563\\
662.778701179284	0.908463876188746\\
664.551863772882	0.910476460752537\\
666.338736965334	0.912510445781\\
668.128711522152	0.9145534886006\\
669.923726592657	0.916607213434235\\
671.725615294753	0.918672883107671\\
673.528675148101	0.920742874516434\\
675.310698896807	0.922790358181033\\
677.075714232289	0.924818494641417\\
678.827325148277	0.926829905680988\\
680.568492320662	0.928826423234775\\
682.301708943003	0.930809293033661\\
684.029123483751	0.932779317656725\\
685.752624928984	0.93473695646996\\
687.473904867979	0.936682398730719\\
689.194503684603	0.938615618086102\\
690.915845529427	0.940536413780941\\
692.639266647147	0.942444443747422\\
694.366037274631	0.944339249838331\\
696.097380722517	0.946220279244995\\
697.834489328548	0.948086901731772\\
699.578538029954	0.949938423521075\\
701.330698304893	0.951774100730879\\
703.092149293825	0.953593148896\\
704.8640904991	0.955394753160975\\
706.647753326961	0.957178076128122\\
708.444414039504	0.958942265867502\\
710.255406697758	0.960686462439868\\
712.082138615965	0.962409805166942\\
713.926107186685	0.964111439139951\\
715.788919190887	0.965790521628354\\
717.672314421463	0.967446229468099\\
719.578193131302	0.969077766192119\\
721.508650871281	0.970684371040205\\
723.466020855369	0.972265328529724\\
725.452928540107	0.973819980670183\\
727.47236243318	0.975347742371491\\
729.527767407267	0.976848121248707\\
731.62317209414	0.978320745164906\\
733.763365207312	0.979765400232999\\
735.954149923374	0.981182087318219\\
738.20272122583	0.982571107323955\\
740.518255785909	0.98393319960323\\
742.912880639819	0.9852697761381\\
745.403384857543	0.986583350825574\\
748.014541500365	0.987878404319201\\
750.701114914338	0.98912516295695\\
753.409816233175	0.990296658937694\\
756.130757943517	0.991389755530654\\
758.857773315495	0.992404146041018\\
761.58572298362	0.993340815655831\\
764.319675705989	0.994204579696234\\
767.064284544192	0.994999739628147\\
769.82394111321	0.995730204156929\\
772.602900688429	0.996399576916625\\
775.321885948516	0.996993879157442\\
777.994947730275	0.997523189747507\\
780.636911380246	0.997996128185814\\
783.258501769735	0.99841925810171\\
785.867935595949	0.998797826493229\\
788.471869901863	0.999136191073324\\
791.075954066315	0.999438074079287\\
793.685174967769	0.999706723008855\\
796.304091660541	0.999945018132502\\
798.937003706805	1.00015554704415\\
801.588082597459	1.00034065822191\\
804.261481362966	1.00050250041153\\
806.961434318712	1.00064305223712\\
809.692353066343	1.00076414472032\\
812.458930364115	1.00086747880407\\
815.26625690306	1.0009546389543\\
818.119960291442	1.0010271037401\\
821.026381421749	1.00108625405792\\
823.992807363255	1.00113337933802\\
827.027787458028	1.00116968187432\\
830.141587155872	1.00119627938155\\
833.346862749555	1.00121420544405\\
836.659718130802	1.00122440717651\\
840.086478819769	1.00122773778285\\
843.52984283198	1.00122520226515\\
846.997158112128	1.00121800907447\\
850.491649274277	1.00120717065722\\
853.994062255339	1.001193624182\\
857.511904146047	1.00117807209403\\
861.052161566385	1.00116105720871\\
864.621583000596	1.00114298882502\\
868.226866238895	1.00112416322655\\
871.8748220102	1.00110478009743\\
875.572528409004	1.00108495593671\\
879.327488239979	1.00106473519466\\
883.101543229174	1.00104434990235\\
886.879487242804	1.00102386198669\\
890.673132358855	1.00100315247708\\
894.49321853986	1.00098208379069\\
898.349972653889	1.00096050521925\\
902.253545997893	1.00093825724599\\
906.214391548349	1.00091517494332\\
910.243631699019	1.00089109054632\\
914.353457620403	1.00086583523049\\
918.557612274302	1.00083923998603\\
922.872016769796	1.00081113539636\\
927.315639001475	1.00078134987921\\
931.911770982395	1.0007497055476\\
936.690024012066	1.0007160100108\\
941.689657355228	1.0006800406345\\
946.958517132953	1.00064156535984\\
952.27602026242	1.00060240982516\\
957.666095489574	1.00056268668045\\
962.931799918129	1.00052414612981\\
968.12404470104	1.00048668260116\\
973.288891647574	1.00045020845523\\
978.459952924227	1.000414719544\\
983.664715747529	1.00038025363754\\
988.927935334399	1.00034686925598\\
994.273717213273	1.00031463419299\\
1000	1.00028211787671\\
};
\addlegendentry{$\text{y}_\text{2}$};

\end{axis}
\end{tikzpicture}%
}
  \caption{Step responses in outputs $y_1$ (\texttt{blue}) and $y_2$
    \texttt{red}.}
  \label{fig:non_minimum_phase_step_23}
\end{figure}

Tables \ref{tbl:minimum_phase_overshoot} and \ref{tbl:minimum_phase_risetime}
feature the overshoot and rise time performance metrics with regard to the
step responses of the system. The large overshoot figures are in line with the
high sensitivity magnitude. It is evident that all inputs are coupled to all
outputs, but the applied control actions result in adequate, but slow,
performance, and certainly worse than in the minimum phase case.

Figure \ref{fig:non_minimum_phase_bode_l_21} shows the frequency response of
$L_{11}(j\omega) = G_{12}(j\omega) f_2(j\omega)$ (\texttt{blue}) and
$L_{22}(j\omega) = G_{21}(j\omega) f_1(j\omega)$ (\texttt{red}), where $L = GF$.
The phase margin is $60^{\circ}$ at the required gain crossover frequency
$\omega_c^{nmp} = 0.02$ rad/s.

\noindent\makebox[\linewidth][c]{%
\begin{minipage}{\linewidth}
  \begin{minipage}{0.45\linewidth}
    \begin{table}[H]\centering
        \begin{tabular}{r|c|c}
        Input/Output & $y_1$    & $y_2$    \\ \hline
          $u_1$          & $0\%$  & $\underline{41.9\%}$ \\
          $u_2$          & $\underline{36.3\%}$ & $0\%$  \\
        \end{tabular}
        \caption{Overshoot figures regarding the step response depicted in figure
          \ref{fig:non_minimum_phase_step_23}. Underlined percentages reference
          control-coupled inputs/outputs.}
        \label{tbl:non_minimum_phase_overshoot}
    \end{table}
  \end{minipage}
  \hfill
  \begin{minipage}{0.45\linewidth}
    \begin{table}[H]\centering
        \begin{tabular}{r|c|c}
        Input/Output & $y_1$    & $y_2$    \\ \hline
          $u_1$          & $137.1$ & ~  \\
          $u_2$          & ~  & $135$ \\
        \end{tabular}
        \caption{Rise times in seconds regarding the step responses depicted in
          figure \ref{fig:non_minimum_phase_step_23}.}
        \label{tbl:non_minimum_phase_risetime}
    \end{table}
  \end{minipage}
\end{minipage}
}


\begin{figure}[H]\centering
\scalebox{1}{% This file was created by matlab2tikz.
%
%The latest updates can be retrieved from
%  http://www.mathworks.com/matlabcentral/fileexchange/22022-matlab2tikz-matlab2tikz
%where you can also make suggestions and rate matlab2tikz.
%
\definecolor{mycolor1}{rgb}{0.00000,0.44700,0.74100}%
\definecolor{mycolor2}{rgb}{0.85000,0.32500,0.09800}%
%
\begin{tikzpicture}

\begin{axis}[%
width=4.008in,
height=1.551in,
at={(0.818in,1.941in)},
scale only axis,
separate axis lines,
every outer x axis line/.append style={white!40!black},
every x tick label/.append style={font=\color{white!40!black}},
xmode=log,
xmin=0.001,
xmax=10,
xtick={0.001,0.01,0.1,1,10},
xticklabels={\empty},
xminorticks=true,
every outer y axis line/.append style={white!40!black},
every y tick label/.append style={font=\color{white!40!black}},
ymin=-150,
ymax=50,
ylabel={Magnitude (dB)},
axis background/.style={fill=white}
]
\addplot [color=mycolor1,solid,forget plot]
  table[row sep=crcr]{%
1e-20	366.843278877946\\
1.02116793351809e-18	326.66133550334\\
1.02116793351809e-13	226.66133550334\\
1.02116793351809e-09	146.66133550334\\
1.02116793351809e-06	86.6613355010579\\
0.000102116793351809	46.6613126871306\\
0.00102116793351809	26.659054268836\\
0.00119182758858131	25.3159029462818\\
0.00139100823114357	23.9724526287793\\
0.00162347634645073	22.6285951688837\\
0.00189479500442508	21.2841833711711\\
0.00221145698651126	19.9390169526425\\
0.00258104016095047	18.5928235056864\\
0.00301238882468553	17.2452327277494\\
0.00351582535149261	15.895741615631\\
0.00410339721117793	14.5436676030048\\
0.00478916527112317	13.1880857343008\\
0.00558954027936968	11.8277449317093\\
0.00652367558143778	10.4609573023994\\
0.00761392546877697	9.08545345021297\\
0.00888638012733827	7.69819632199388\\
0.0103714899878364	6.29514702550444\\
0.0121047944186931	4.87097965441616\\
0.0141277721996231	3.41875051513163\\
0.0164888341280886	1.92954298330259\\
0.0192444815121576	0.392135138649088\\
0.0224606582730361	-1.20722623494796\\
0.0262143290137176	-2.88482041435379\\
0.0305953208176631	-4.65906180202648\\
0.0357084728526103	-6.54957934923987\\
0.0416761452205291	-8.57570258711214\\
0.0486411470916674	-10.7544446883717\\
0.0567701541942941	-13.0982546284767\\
0.066257697442255	-15.6129342418999\\
0.0773308181500527	-18.2961197082237\\
0.0902545012369055	-21.136625290711\\
0.105338016438884	-24.1148025289457\\
0.122942319277272	-27.2039180872329\\
0.143488688891765	-30.3724011924619\\
0.167468809445862	-33.5866742713517\\
0.195456536357159	-36.8141913725548\\
0.228121628923908	-40.0262731429083\\
0.251885765359407	-42.0670788324355\\
0.266245777975953	-43.2003282655562\\
0.310741312099194	-46.321109465966\\
0.362673030082189	-49.3808205947787\\
0.423283682045499	-52.3781663784736\\
0.494023709029018	-55.3167060178168\\
0.576585952719417	-58.2029944996215\\
0.672946165937616	-61.0449256894074\\
0.785410293321022	-63.8505049040997\\
0.916669653649231	-66.627090880909\\
1.06986534435188	-69.38102836107\\
1.24866340942835	-72.1175506854601\\
1.4573425695827	-74.8408403640661\\
1.70089661399641	-77.5541646990646\\
1.98515390402193	-80.2600340123744\\
2.31691684857564	-82.9603537554587\\
2.70412468894116	-85.6565574531866\\
3.15604348849917	-88.3497165791783\\
3.68348787392575	-91.0406281745193\\
4.29907983422947	-93.7298830598544\\
5.01755077080812	-96.4179180645376\\
5.85609402672315	-99.1050555526951\\
6.83477633138115	-101.791533087283\\
7.97701800668447	-104.477525559585\\
9.31015342620724	-107.163161626389\\
10.8660851394449	-109.848535880637\\
12.6820473146343	-112.533717842314\\
126.820473146343	-152.535170710779\\
12682.0473146343	-232.535185388172\\
12682047.3146343	-352.53518538964\\
126820473146.343	-512.53518538964\\
1.26820473146343e+16	-712.53518538964\\
1e+20	-868.407610624228\\
};
\addplot [color=black,solid,forget plot]
  table[row sep=crcr]{%
0.0975669563409678	0\\
0.0975669563409678	-22.6222742114512\\
};
\addplot [color=black,dotted,forget plot]
  table[row sep=crcr]{%
0.0975669563409678	-150\\
0.0975669563409678	0\\
};
\addplot [color=black,dotted,forget plot]
  table[row sep=crcr]{%
0.0199999146718952	-150\\
0.0199999146718952	0\\
};
\addplot [color=black,dotted,forget plot]
  table[row sep=crcr]{%
0.001	0\\
10	0\\
};
\addplot [color=mycolor2,solid,forget plot]
  table[row sep=crcr]{%
1e-20	366.910966538527\\
9.3840342877527e-19	327.463174819476\\
9.38403428775266e-14	227.463174819476\\
9.38403428775266e-10	147.463174819476\\
9.38403428775266e-07	87.463174817364\\
9.38403428775266e-05	47.4631537038019\\
0.000938403428775266	27.4610636195102\\
0.00109596264707322	26.1121720065442\\
0.00127997627347479	24.763000903316\\
0.00149488603925838	23.4134487133542\\
0.00174587944845496	22.0633769767389\\
0.00203901499411258	20.7125970524613\\
0.00238136839854277	19.3608520392404\\
0.00278120340750425	18.0077922749833\\
0.00324817126096347	16.6529422130667\\
0.00379354365526856	15.2956557863275\\
0.00443048481999058	13.9350565267211\\
0.00517436927683843	12.5699577292159\\
0.00604315295072975	11.1987569158844\\
0.00705780659091796	9.81929797853085\\
0.00824282196411888	8.42869408818214\\
0.00962680303815519	7.02310556090807\\
0.0112431564261428	5.59747071571256\\
0.0131308977571997	4.14519638831881\\
0.0153355934378995	2.65783075375648\\
0.0179104605367592	1.12476670630552\\
0.020917651340837	-0.466941125286339\\
0.0244297535911375	-2.13252563625745\\
0.0285315426095927	-3.88908160643673\\
0.0333220276105578	-5.75464737738708\\
0.0389168415908033	-7.74676861435028\\
0.0454510324853045	-9.88067202480063\\
0.0530823229619022	-12.1673267304312\\
0.0619949175399422	-14.6117272955039\\
0.0724039489293387	-17.2116631259676\\
0.0845606709160433	-19.9571227506646\\
0.0987585231400815	-22.8304309594155\\
0.115340214158111	-25.807236100725\\
0.134705993761866	-28.858431705368\\
0.157323314230173	-31.9529103584174\\
0.183738113718385	-35.0607572758343\\
0.214587994144324	-38.156257678217\\
0.250617611659289	-41.220063757588\\
0.258756792821962	-41.8457885665419\\
0.292696651200174	-44.2400909101174\\
0.341840818993461	-47.2111068931737\\
0.399236359729319	-50.1333598672961\\
0.466268690202757	-53.0107860003822\\
0.544555840582243	-55.8492926748271\\
0.635987510513054	-58.6554172826568\\
0.74277068279374	-61.4354487885571\\
0.867482895651225	-64.194956218606\\
1.01313445950371	-66.9386100619878\\
1.18324112000308	-69.670182946052\\
1.38190892130149	-72.3926422869772\\
1.61393331797638	-75.1082779949845\\
1.88491492798313	-77.8188328919355\\
2.20139472068674	-80.5256201431258\\
2.57101190314879	-83.2296219481472\\
3.0026883157377	-85.9315690291301\\
3.50684378801413	-88.6320028906098\\
4.09564765316374	-91.3313236667566\\
4.78331249204707	-94.0298264081688\\
5.58643719727685	-96.7277283443285\\
6.52440763822281	-99.4251892290151\\
7.61986459821836	-102.122326450444\\
8.89925027906397	-104.819226214067\\
10.3934465643835	-107.515951797789\\
103.934465643835	-147.517254814086\\
10393.4465643835	-227.517267977888\\
10393446.5643835	-347.517267979204\\
103934465643.835	-507.517267979204\\
1.03934465643835e+16	-707.517267979204\\
1e+20	-866.846884476588\\
};
\addplot [color=black,solid,forget plot]
  table[row sep=crcr]{%
0.112525164673436	0\\
0.112525164673436	-25.3275560557658\\
};
\addplot [color=black,dotted,forget plot]
  table[row sep=crcr]{%
0.112525164673436	-150\\
0.112525164673436	0\\
};
\addplot [color=black,dotted,forget plot]
  table[row sep=crcr]{%
0.0199999191691064	-150\\
0.0199999191691064	0\\
};
\addplot [color=black,dotted,forget plot]
  table[row sep=crcr]{%
0.001	0\\
10	0\\
};
\end{axis}

\begin{axis}[%
width=4.008in,
height=1.376in,
at={(0.818in,0.44in)},
scale only axis,
separate axis lines,
every outer x axis line/.append style={white!40!black},
every x tick label/.append style={font=\color{white!40!black}},
xmode=log,
xmin=0.001,
xmax=10,
xminorticks=true,
xlabel={Frequency [rad/s]},
every outer y axis line/.append style={white!40!black},
every y tick label/.append style={font=\color{white!40!black}},
ymin=-226.35,
ymax=-88.65,
ytick={-225, -180, -135,  -90},
ylabel={Phase (deg)},
axis background/.style={fill=white}
]
\addplot [color=mycolor1,solid,forget plot]
  table[row sep=crcr]{%
1e-20	-90\\
1.02116793351809e-18	-90\\
1.02116793351809e-13	-90.0000000001597\\
1.02116793351809e-09	-90.0000015969701\\
1.02116793351809e-06	-90.001596970135\\
0.000102116793351809	-90.1596968298085\\
0.00102116793351809	-91.5967864608049\\
0.00119182758858131	-91.8635670342848\\
0.00139100823114357	-92.1748867604949\\
0.00162347634645073	-92.5381621504403\\
0.00189479500442508	-92.9620333050535\\
0.00221145698651126	-93.4565590777717\\
0.00258104016095047	-94.03343939367\\
0.00301238882468553	-94.706266091522\\
0.00351582535149261	-95.490802067919\\
0.00410339721117793	-96.405285642662\\
0.00478916527112317	-97.4707521970264\\
0.00558954027936968	-98.7113571750345\\
0.00652367558143778	-100.154671937428\\
0.00761392546877697	-101.831904645518\\
0.00888638012733827	-103.777969811964\\
0.0103714899878364	-106.031289916028\\
0.0121047944186931	-108.633159458139\\
0.0141277721996231	-111.626439235324\\
0.0164888341280886	-115.053289205894\\
0.0192444815121576	-118.951621902957\\
0.0224606582730361	-123.350018680443\\
0.0262143290137176	-128.26107250783\\
0.0305953208176631	-133.673568216892\\
0.0357084728526103	-139.54456764062\\
0.0416761452205291	-145.79314638135\\
0.0486411470916674	-152.297863356514\\
0.0567701541942941	-158.899672627183\\
0.066257697442255	-165.410888073114\\
0.0773308181500527	-171.629435401206\\
0.0902545012369055	-177.356544675292\\
0.105338016438884	-182.415463179526\\
0.122942319277272	-186.668581596582\\
0.143488688891765	-190.030511579222\\
0.167468809445862	-192.475200912088\\
0.195456536357159	-194.036034482319\\
0.228121628923908	-194.798822200765\\
0.251885765359407	-194.924810218178\\
0.266245777975953	-194.888586105737\\
0.310741312099194	-194.452230965412\\
0.362673030082189	-193.640278026768\\
0.423283682045499	-192.591178801902\\
0.494023709029018	-191.420820725646\\
0.576585952719417	-190.218033219951\\
0.672946165937616	-189.045153853847\\
0.785410293321022	-187.94178253453\\
0.916669653649231	-186.929839920279\\
1.06986534435188	-186.018576986513\\
1.24866340942835	-185.208818963248\\
1.4573425695827	-184.496216292321\\
1.70089661399641	-183.873558139591\\
1.98515390402193	-183.332322853653\\
2.31691684857564	-182.863660621677\\
2.70412468894116	-182.458978996233\\
3.15604348849917	-182.110263564475\\
3.68348787392575	-181.810229089232\\
4.29907983422947	-181.552366522173\\
5.01755077080812	-181.330929133506\\
5.85609402672315	-181.140885490542\\
6.83477633138115	-180.977856561263\\
7.97701800668447	-180.838047366232\\
9.31015342620724	-180.718179212546\\
10.8660851394449	-180.615425789901\\
12.6820473146343	-180.527354718965\\
126.820473146343	-180.052749884955\\
12682.0473146343	-180.000527500306\\
12682047.3146343	-180.0000005275\\
126820473146.343	-180.000000000053\\
1.26820473146343e+16	-180\\
1e+20	-180\\
};
\addplot [color=black,solid,forget plot]
  table[row sep=crcr]{%
0.0199999146718952	-180\\
0.0199999146718952	-119.999882110632\\
};
\addplot [color=black,dotted,forget plot]
  table[row sep=crcr]{%
0.0199999146718952	-180\\
0.0199999146718952	-88.65\\
};
\addplot [color=black,dotted,forget plot]
  table[row sep=crcr]{%
0.0975669563409678	-180\\
0.0975669563409678	-88.65\\
};
\addplot [color=black,dotted,forget plot]
  table[row sep=crcr]{%
0.001	-180\\
10	-180\\
};
\addplot [color=mycolor2,solid,forget plot]
  table[row sep=crcr]{%
1e-20	-90\\
9.3840342877527e-19	-90\\
9.38403428775266e-14	-90.0000000001479\\
9.38403428775266e-10	-90.0000014787868\\
9.38403428775266e-07	-90.0014787868209\\
9.38403428775266e-05	-90.1478785100596\\
0.000938403428775266	-91.4786148110529\\
0.00109596264707322	-91.7268031578664\\
0.00127997627347479	-92.0166193398645\\
0.00149488603925838	-92.3550270597641\\
0.00174587944845496	-92.7501440446328\\
0.00203901499411258	-93.211426835347\\
0.00238136839854277	-93.7498813229963\\
0.00278120340750425	-94.3783002727857\\
0.00324817126096347	-95.1115275125421\\
0.00379354365526856	-95.9667456510261\\
0.00443048481999058	-96.9637794102662\\
0.00517436927683843	-98.1253988435973\\
0.00604315295072975	-99.4775943625973\\
0.00705780659091796	-101.049776612027\\
0.00824282196411888	-102.87482643363\\
0.00962680303815519	-104.988881234366\\
0.0112431564261428	-107.430693419338\\
0.0131308977571997	-110.240338222788\\
0.0153355934378995	-113.456996349007\\
0.0179104605367592	-117.115522767846\\
0.020917651340837	-121.24159127533\\
0.0244297535911375	-125.845445937751\\
0.0285315426095927	-130.914745981256\\
0.0333220276105578	-136.407613019681\\
0.0389168415908033	-142.24755055382\\
0.0454510324853045	-148.322021864687\\
0.0530823229619022	-154.485859952482\\
0.0619949175399422	-160.569528223497\\
0.0724039489293387	-166.391235404404\\
0.0845606709160433	-171.771593838333\\
0.0987585231400815	-176.549655520999\\
0.115340214158111	-180.598965272492\\
0.134705993761866	-183.841429134362\\
0.157323314230173	-186.256010091237\\
0.183738113718385	-187.87954553873\\
0.214587994144324	-188.798672777472\\
0.250617611659289	-189.134371510217\\
0.258756792821962	-189.143443147418\\
0.292696651200174	-189.022873087006\\
0.341840818993461	-188.597563889966\\
0.399236359729319	-187.975595790512\\
0.466268690202757	-187.2507853762\\
0.544555840582243	-186.492224015084\\
0.635987510513054	-185.746763774981\\
0.74277068279374	-185.043359108471\\
0.867482895651225	-184.397731521508\\
1.01313445950371	-183.816488304791\\
1.18324112000308	-183.300363895419\\
1.38190892130149	-182.846577442892\\
1.61393331797638	-182.450447742295\\
1.88491492798313	-182.106444496924\\
2.20139472068674	-181.808839332885\\
2.57101190314879	-181.552086052413\\
3.0026883157377	-181.331024737066\\
3.50684378801413	-181.140975272456\\
4.09564765316374	-180.977764054254\\
4.78331249204707	-180.837712224967\\
5.58643719727685	-180.717603334724\\
6.52440763822281	-180.614641426393\\
7.61986459821836	-180.526406102113\\
8.89925027906397	-180.450808318761\\
10.3934465643835	-180.386048916258\\
103.934465643835	-180.038618575503\\
10393.4465643835	-180.000386187137\\
10393446.5643835	-180.000000386187\\
103934465643.835	-180.000000000039\\
1.03934465643835e+16	-180\\
1e+20	-180\\
};
\addplot [color=black,solid,forget plot]
  table[row sep=crcr]{%
0.0199999191691064	-180\\
0.0199999191691064	-119.999889947511\\
};
\addplot [color=black,dotted,forget plot]
  table[row sep=crcr]{%
0.0199999191691064	-180\\
0.0199999191691064	-88.65\\
};
\addplot [color=black,dotted,forget plot]
  table[row sep=crcr]{%
0.112525164673436	-180\\
0.112525164673436	-88.65\\
};
\addplot [color=black,dotted,forget plot]
  table[row sep=crcr]{%
0.001	-180\\
10	-180\\
};
\end{axis}
\end{tikzpicture}%
}
\caption{$L_{11}(j\omega) = G_{12}(j\omega) f_2(j\omega)$ (\texttt{blue}) and
  $L_{22}(j\omega) = G_{21}(j\omega) f_1(j\omega)$ (\texttt{red}), where $L = GF$.
  The phase margin is $60^{\circ}$ and the gain crossover frequency is
  $\omega_c^{nmp} = 0.02$ rad/s.}
\label{fig:non_minimum_phase_bode_l_21}
\end{figure}
